\documentclass{amsart}
\usepackage[utf8]{inputenc}
\usepackage{amssymb}
\usepackage{hyperref}
\usepackage{graphicx}

\theoremstyle{definition}
\newtheorem{mydef}{Definition}[section]
\newtheorem{lem}[mydef]{Lemma}
\newtheorem{thm}[mydef]{Theorem}
\newtheorem{cor}[mydef]{Corollary}
\newtheorem{claim}[mydef]{Claim}
\newtheorem{question}[mydef]{Question}
\newtheorem{hypothesis}[mydef]{Hypothesis}
\newtheorem{prop}[mydef]{Proposition}
\newtheorem{defin}[mydef]{Definition}
\newtheorem{example}[mydef]{Example}
\newtheorem{remark}[mydef]{Remark}
\newtheorem{notation}[mydef]{Notation}
\newtheorem{fact}[mydef]{Fact}

\newcommand{\bA}{\mathbf{A}}
\newcommand{\bB}{\mathbf{B}}
\newcommand{\bC}{\mathbf{C}}
\newcommand{\bD}{\mathbf{D}}

\newcommand{\ba}{\bar{a}}
\newcommand{\bb}{\bar{b}}

\newcommand{\bx}{\bar{x}}
\newcommand{\by}{\bar{y}}

\newcommand{\seq}[1]{\langle #1 \rangle}
\newcommand{\fct}[2]{{}^{#1}{#2}}
\newcommand{\rest}{\upharpoonright}

\newcommand{\proves}{\vdash}
\newcommand{\imp}{\rightarrow}
\newcommand{\impp}{\leftrightarrow}
\newcommand{\PA}{\operatorname{PA}}

\title[Math 141a, Fall 2018: assignment 10]{Math 141a - Mathematical logic I, Fall 2018 \\ Assignment 10}

%% Include only sections, not subsections, in the table of content.
\setcounter{tocdepth}{1}

%% \author{Sebastien Vasey}
%% \email{sebv@cmu.edu}
%% \address{Department of Mathematical Sciences, Carnegie Mellon University, Pittsburgh, Pennsylvania, USA}
\date{\today}

\begin{document}

%% No indentation at the start of each paragraph
%\parindent 0pt

\maketitle

\textbf{Due Friday, November 16 at the beginning of class} (please submit your assignment on Canvas). Make sure to include your full name \emph{and the list of your collaborators} (if any) with your assignment. You may discuss problems with others, but you may \emph{not} keep a written record of your discussions. Please refer to the syllabus for details.

With regards to answering these problems, imagine that you are writing an answer to teach someone else in the class how to do the problem. In particular, you must give a complete outline for how you arrived at your answer. It is not sufficient to simply state a number or formula without providing the steps and reasoning that you used to produce the answer.

\begin{enumerate}
  \item Assume $I$ is a non-empty set. Recall that a \emph{partition} of $I$ is a non-empty collection $P$ of pairwise disjoint subsets of $I$ such that the union of all sets in $P$ is $I$ (we allow the empty set to be in $P$). Prove that the following are equivalent, for a collection $U$ of subsets of $I$:

  \begin{enumerate}
  \item $U$ is an ultrafilter on $I$.
  \item $|U \cap P| = 1$ for any partition $P$ of $I$ with $|P| \le 3$.
  \item $I \in U$, $U$ is closed under taking supersets ($A \in U$, $A \subseteq B \subseteq I$ implies $B \in U$), and for any two disjoint subsets $A$ and $B$ of $I$, if $A \cup B \in U$ then exactly one of $A$ or $B$ is in $U$.
  \end{enumerate}

\item Assume that $(A, V, F)$ is a voting system with at least three but finitely-many candidates (but possibly infinitely-many voters). Assume this voting system satisfies unanimity and independence of irrelevant alternatives. Show that there exists an ultrafilter $U$ on $V$ such that for any preference profile $(C_v)_{v \in V}$, $C = F ((C_v)_{v \in V})$ is given by, for $a, b \in A$, $a <^C b$ if and only if $\{v \in V \mid a <^{C_v} b\} \in U$.
\item (The infinite pigeonhole principle) Let $\lambda$ be an infinite cardinal.

  \begin{enumerate}
  \item Show that for any finite cardinal $\mu$ and any function $f: \lambda \to \mu$ there exists $A \subseteq \lambda$ with $|A| = \lambda$ such that $f$ is constant on $A$.
  \item Show that for any \emph{infinite} cardinal $\mu < \lambda$ and any $\chi < \lambda$, there exists $A \subseteq \lambda$ with $|A| \ge \chi^+$ such that $f$ is constant on $A$. (\emph{In general, we cannot find such an $A$ of cardinality $\lambda$, see the extra credit problem.})
  \end{enumerate}
\item Prove the De Bruijn-Erdős theorem: for any natural number $k < \omega$, a graph is $k$-colorable if and only if all of its finite subgraphs are $k$-colorable.
\item (Extra credit)
  \begin{enumerate}
  \item Give an example of infinite cardinals $\mu < \lambda$ and a function $f: \lambda \to \mu$ such that $f$ is not constant on any set of cardinality $\lambda$.
  \item Give an example of a function $f: [\mathbb{R}]^2 \to 2$ that has no uncountable homogeneous set. \emph{Note: this shows that we cannot replace ``infinite homogeneous set'' by ``homogeneous set of the cardinality of the domain'' in the statement of the infinite Ramsey theorem.} \emph{Hint: well-order the real numbers and compare the well-ordering with the usual ordering.}
  \end{enumerate}
\end{enumerate}

\bibliographystyle{amsalpha}
\bibliography{notes}

\end{document}
