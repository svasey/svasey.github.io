\documentclass{amsart}
\usepackage{amssymb}
\usepackage{hyperref}
\usepackage{graphicx}

\theoremstyle{definition}
\newtheorem{mydef}{Definition}[section]
\newtheorem{lem}[mydef]{Lemma}
\newtheorem{thm}[mydef]{Theorem}
\newtheorem{cor}[mydef]{Corollary}
\newtheorem{claim}[mydef]{Claim}
\newtheorem{question}[mydef]{Question}
\newtheorem{hypothesis}[mydef]{Hypothesis}
\newtheorem{prop}[mydef]{Proposition}
\newtheorem{defin}[mydef]{Definition}
\newtheorem{example}[mydef]{Example}
\newtheorem{remark}[mydef]{Remark}
\newtheorem{notation}[mydef]{Notation}
\newtheorem{fact}[mydef]{Fact}

\newcommand{\bA}{\mathbf{A}}
\newcommand{\bB}{\mathbf{B}}
\newcommand{\bC}{\mathbf{C}}
\newcommand{\bD}{\mathbf{D}}

\newcommand{\ba}{\bar{a}}
\newcommand{\bb}{\bar{b}}

\newcommand{\bx}{\bar{x}}
\newcommand{\by}{\bar{y}}

\newcommand{\seq}[1]{\langle #1 \rangle}
\newcommand{\fct}[2]{{}^{#1}{#2}}
\newcommand{\rest}{\upharpoonright}

\title[Math 141a, Fall 2018: assignment 5]{Math 141a - Mathematical logic I, Fall 2018 \\ Assignment 5}

%% Include only sections, not subsections, in the table of content.
\setcounter{tocdepth}{1}

%% \author{Sebastien Vasey}
%% \email{sebv@cmu.edu}
%% \address{Department of Mathematical Sciences, Carnegie Mellon University, Pittsburgh, Pennsylvania, USA}
\date{\today}

\begin{document}

%% No indentation at the start of each paragraph
%\parindent 0pt

\maketitle

\textbf{Due Friday, October 12 at the beginning of class} (please submit your assignment on Canvas). Make sure to include your full name \emph{and the list of your collaborators} (if any) with your assignment. You may discuss problems with others, but you may \emph{not} keep a written record of your discussions. Please refer to the syllabus for details.

With regards to answering these problems, imagine that you are writing an answer to teach someone else in the class how to do the problem. In particular, you must give a complete outline for how you arrived at your answer. It is not sufficient to simply state a number or formula without providing the steps and reasoning that you used to produce the answer.

\begin{enumerate}
\item Show that there are only countably-many formulas (in the language for an $m$-ary relation, for a fixed $m < \omega$).
\item Let $f (\bx)$ be a formula (in the language for a binary relation), with $\bx = (x_1, \ldots, x_n)$, $n \ge 1$.
  \begin{enumerate}
  \item Show that there is a \emph{quantifier-free} formula $g (\bx)$ so that for any tuple $\ba$ of rationals, $(\mathbb{Q}, <) \models f (\ba)$ if and only if $(\mathbb{Q}, <) \models g (\ba)$. \emph{Note: the proof will in fact give an algorithm that takes as input a formula $f (\bx)$ and a tuple $\ba$ of rationals and outputs whether or not $(\mathbb{Q}, <) \models f (\ba)$.}

    \emph{Hint: to see one idea, first solve the problem when $f (x_1, x_2)$ is $(\exists x) (x_1 < x \land x < x_2)$. Then remember from the previous assignment that every formula can be put in prenex form, and every quantifier-free formula has a disjunctive form.}
  \item Let $g (\bx)$ be as in the previous part. Let $T$ be the theory of non-empty dense chains without endpoints. Show that:
    
    $$T
    \models \forall x_1 \forall x_2 \ldots \forall x_n \left(f (\bx) \leftrightarrow g (\bx)\right)
    $$

    \emph{Note: when such a phenomenon happens, we say that $T$ has quantifier elimination.}
  \end{enumerate}

\item Assume that $T$ is a theory (in the language for an $m$-ary relation). Prove that if any two \emph{countable} models of $T$ are elementarily equivalent, then $T$ is complete.
\item
  
  \begin{enumerate}
  \item For each of the following properties, give a sentence (in the language for a binary relation) which expresses it:
    \begin{enumerate}
    \item $r$ is (the graph of) a function (from the universe of $r$ to itself).
    \item $r$ is a surjection.
    \item $r$ is an injection.
    \item $r$ is a bijection.
    \end{enumerate}

  \item Give a set of axioms whose models are exactly the bijections with infinite domain. Is this set complete? 
  \item Give a set of axioms $A$ in the language for a binary relation so that the models of $A$ are exactly the bijections $R$ with non-empty domain such that $R^n (x) \neq x$ for any $x$ and any nonzero $n < \omega$. Here, we use $R^n$ to denote the composition of $R$ with itself $n$ times.
  \item Show that $A$ is consistent, by giving both a countable and an uncountable model for $A$.
  \item Let $(E, R)$ be a model of $A$. We write $R (x)$ for the output of the function $R$ at $x$, $R^{n}$ for the composition of $R$ with itself $n$ times ($n \ge 1$), and $R^{-n}$ for the composition of the inverse $R^{-1}$ with itself $n$ times. When $n = 0$, $R^0$ is notation for the identity function on $E$. Show that the relation $\approx$ on $E$ defined  $x \approx y$ if and only if there exists $n \in \mathbb{Z}$ so that $R^n (x) = y$ is an equivalence relation.
  \item Let $(E, R)$ be a model of $A$. Let $C \subseteq E$ be an $\approx$-equivalence class. For $x, y \in C$, show that the $n \in \mathbb{Z}$ such that $y = R^n (x)$ is unique. Deduce that defining  $x < y$ to hold if and only if there exists a strictly positive $n$ such that $y = R^n (x)$ gives a total ordering on $C$. Show moreover that $(\mathbb{Z}, <) \cong (C, <)$.
  \item Is $A$ complete? \emph{Hint: change the structure a little bit to make it look like a non-empty discrete chain without endpoints, then use Theorem 1.8 in Poizat. You do not need to give full details, but explain the intuition carefully.}
  \end{enumerate}
\item
  \begin{enumerate}
  \item Let $(C, <)$ and $(C', <)$ be two discrete chains with minimums but no maximums. Let $a_0$ be the minimum of $C$ and $a_0'$ be the minimum of $C'$. Let $a_1 < a_2 < \ldots < a_n$ be an increasing sequence in $C$ and let $a_1' < a_2' < \ldots < a_n'$ be an increasing sequence in $C'$. Fix $p < \omega$ and assume that for any $i < n$, at least one of the following is true:

    \begin{itemize}
    \item $d (a_i, a_{i + 1}) = d (a_i', a_{i + 1}')$. \emph{(See right before 1.8 in Poizat for the definition of $d$)}.
    \item $d (a_i, a_{i + 1}) \ge 2^{p} - 1$ and $d (a_{i}', a_{i + 1}') \ge 2^p - 1$.
    \end{itemize}

    Show that the map sending $a_i$ to $a_i'$ for each $i \le n$ is a $p$-isomorphism. \emph{Hint: look at the proof of Theorem 1.8 in Poizat.}
    
  \item Deduce that $(\mathbb{N}, <) \preceq (\mathbb{N}, <) + (\mathbb{Z}, <)$. 
  \item Is there a set of axioms (in the langugage for a binary relation) whose models are exactly the well-orderings? 
  \end{enumerate}
\end{enumerate}

\bibliographystyle{amsalpha}
\bibliography{notes}

\end{document}
