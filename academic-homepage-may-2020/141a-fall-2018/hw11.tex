\documentclass{amsart}
\usepackage[utf8]{inputenc}
\usepackage{amssymb}
\usepackage{hyperref}
\usepackage{graphicx}

\theoremstyle{definition}
\newtheorem{mydef}{Definition}[section]
\newtheorem{lem}[mydef]{Lemma}
\newtheorem{thm}[mydef]{Theorem}
\newtheorem{cor}[mydef]{Corollary}
\newtheorem{claim}[mydef]{Claim}
\newtheorem{question}[mydef]{Question}
\newtheorem{hypothesis}[mydef]{Hypothesis}
\newtheorem{prop}[mydef]{Proposition}
\newtheorem{defin}[mydef]{Definition}
\newtheorem{example}[mydef]{Example}
\newtheorem{remark}[mydef]{Remark}
\newtheorem{notation}[mydef]{Notation}
\newtheorem{fact}[mydef]{Fact}

\newcommand{\bA}{\mathbf{A}}
\newcommand{\bB}{\mathbf{B}}
\newcommand{\bC}{\mathbf{C}}
\newcommand{\bD}{\mathbf{D}}

\newcommand{\ba}{\bar{a}}
\newcommand{\bb}{\bar{b}}

\newcommand{\bx}{\bar{x}}
\newcommand{\by}{\bar{y}}

\newcommand{\seq}[1]{\langle #1 \rangle}
\newcommand{\fct}[2]{{}^{#1}{#2}}
\newcommand{\rest}{\upharpoonright}

\newcommand{\proves}{\vdash}
\newcommand{\imp}{\rightarrow}
\newcommand{\impp}{\leftrightarrow}
\newcommand{\PA}{\operatorname{PA}}

\newcommand{\dom}{\operatorname{dom}}
\newcommand{\im}{\operatorname{im}}

\newcommand{\cl}{\operatorname{cl}}
\newcommand{\acl}{\operatorname{acl}}

\title[Math 141a, Fall 2018: assignment 11]{Math 141a - Mathematical logic I, Fall 2018 \\ Assignment 11}

%% Include only sections, not subsections, in the table of content.
\setcounter{tocdepth}{1}

%% \author{Sebastien Vasey}
%% \email{sebv@cmu.edu}
%% \address{Department of Mathematical Sciences, Carnegie Mellon University, Pittsburgh, Pennsylvania, USA}
\date{\today}

\begin{document}

%% No indentation at the start of each paragraph
%\parindent 0pt

\maketitle

\textbf{Due Friday, November 30 at the beginning of class} (please submit your assignment on Canvas). Make sure to include your full name \emph{and the list of your collaborators} (if any) with your assignment. You may discuss problems with others, but you may \emph{not} keep a written record of your discussions. Please refer to the syllabus for details.

\textbf{Special instructions:} Do \textbf{exactly five} of the problems below. Please make sure to clearly mark which problems you have chosen. You should attempt the other problems too, but they will not be graded for credit.

With regards to answering these problems, imagine that you are writing an answer to teach someone else in the class how to do the problem. In particular, you must give a complete outline for how you arrived at your answer. It is not sufficient to simply state a number or formula without providing the steps and reasoning that you used to produce the answer.

\begin{enumerate}
\item For a nonzero hyperreal number $\delta$, we say that a hyperreal number $x$ is \emph{infinitesimal relative to $\delta$} if $|x| < r |\delta|$ for any positive real number $r$. Thus an infinitesimal is exactly an infinitesimal relative to a nonzero real number.

  \begin{enumerate}
  \item Assume that $\delta$ is an infinite hyperreal. Show that any real number is infinitesimal relative to $\delta$.
  \item Show that for any nonzero hyperreal $\delta$, there exists a nonzero hyperreal $\epsilon$ that is infinitesimal with respect to $\delta$.
  \end{enumerate}
  
\item Give a proof (using nonstandard analysis) of the intermediate value theorem: if $f: \mathbb{R} \to \mathbb{R}$ is continuous and $a < b$ are real numbers such that $f (a) < 0 < f (b)$, then there exists a real number $x \in [a,b]$ such that $f (x) = 0$.

\item A sequence $(a_n)$ of real numbers is nothing but a function from $\mathbb{N}$ to $\mathbb{R}$. Using the extension principle we can, given such a sequence, make sense of $a_N$ for $N$ any hypernatural number. Thus we say that a sequence $(a_n)$ of real numbers \emph{converges} to a real number $a$ if $a_N \simeq a$ for any infinite hypernatural $N$.

  \begin{enumerate}
  \item Prove that this is equivalent to the standard definition of convergence: for any real number $\epsilon > 0$ there exists $n \in \mathbb{N}$ such that for all $m \ge n$, $|a_m - a| < \epsilon$.
  \item Recall that a sequence $(a_n)$ of real numbers is \emph{Cauchy} if for any real number $\epsilon > 0$ there exists $n \in \mathbb{N}$ such that for any $m_1, m_2 \ge n$, $|a_{m_1} - a_{m_2}| < \epsilon$. Give a ``nonstandard'' definition of a Cauchy sequence, prove it is equivalent to the classical definition, and use your definition to prove that any Cauchy sequence converges.
  \end{enumerate}

\item Prove that for any infinite structure $M$ and any cardinal $\lambda$, $M$ has an elementary extension of cardinality at least $\lambda$.
  
\item Prove that if $F$ is a field and $X$ is a subset of $F$, then $|\acl^F (X)| \le |X| + \aleph_0$.
\item Assume $E$ and $F$ are fields, $s$ is a local isomorphism from $E$ to $F$, $E_0 = \langle \dom (s) \rangle$ is the subfield of $E$ generated by $\dom (s)$, and $F_0 = \langle \im (s) \rangle$ is the subfield of $F$ generated by $\im (s)$. Show that $s$ extends to an isomorphism $f: E_0 \cong F_0$.
\item Using only that for any field $F$, any nonconstant polynomial with coefficients from $F$ has a root in a field extension of $F$, show that any field has an algebraically closed extension. \emph{Use transfinite induction: keep adding roots until you reach an algebraically closed extension!}
\end{enumerate}
\bibliographystyle{amsalpha}
\bibliography{notes}

\end{document}
