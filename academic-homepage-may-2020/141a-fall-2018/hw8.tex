\documentclass{amsart}
\usepackage[utf8]{inputenc}
\usepackage{amssymb}
\usepackage{hyperref}
\usepackage{graphicx}

\theoremstyle{definition}
\newtheorem{mydef}{Definition}[section]
\newtheorem{lem}[mydef]{Lemma}
\newtheorem{thm}[mydef]{Theorem}
\newtheorem{cor}[mydef]{Corollary}
\newtheorem{claim}[mydef]{Claim}
\newtheorem{question}[mydef]{Question}
\newtheorem{hypothesis}[mydef]{Hypothesis}
\newtheorem{prop}[mydef]{Proposition}
\newtheorem{defin}[mydef]{Definition}
\newtheorem{example}[mydef]{Example}
\newtheorem{remark}[mydef]{Remark}
\newtheorem{notation}[mydef]{Notation}
\newtheorem{fact}[mydef]{Fact}

\newcommand{\bA}{\mathbf{A}}
\newcommand{\bB}{\mathbf{B}}
\newcommand{\bC}{\mathbf{C}}
\newcommand{\bD}{\mathbf{D}}

\newcommand{\ba}{\bar{a}}
\newcommand{\bb}{\bar{b}}

\newcommand{\bx}{\bar{x}}
\newcommand{\by}{\bar{y}}

\newcommand{\seq}[1]{\langle #1 \rangle}
\newcommand{\fct}[2]{{}^{#1}{#2}}
\newcommand{\rest}{\upharpoonright}

\newcommand{\proves}{\vdash}
\newcommand{\imp}{\rightarrow}
\newcommand{\impp}{\leftrightarrow}


\title[Math 141a, Fall 2018: assignment 8]{Math 141a - Mathematical logic I, Fall 2018 \\ Assignment 8}

%% Include only sections, not subsections, in the table of content.
\setcounter{tocdepth}{1}

%% \author{Sebastien Vasey}
%% \email{sebv@cmu.edu}
%% \address{Department of Mathematical Sciences, Carnegie Mellon University, Pittsburgh, Pennsylvania, USA}
\date{\today}

\begin{document}

%% No indentation at the start of each paragraph
%\parindent 0pt

\maketitle

\textbf{Due Friday, November 2 at the beginning of class} (please submit your assignment on Canvas). Make sure to include your full name \emph{and the list of your collaborators} (if any) with your assignment. You may discuss problems with others, but you may \emph{not} keep a written record of your discussions. Please refer to the syllabus for details.

With regards to answering these problems, imagine that you are writing an answer to teach someone else in the class how to do the problem. In particular, you must give a complete outline for how you arrived at your answer. It is not sufficient to simply state a number or formula without providing the steps and reasoning that you used to produce the answer.

\begin{enumerate}

\item Assume $A$ is a set of sentences and $\phi, \psi$ are sentences. Prove (without using the completeness theorem) that the following are equivalent:
  \begin{enumerate}
  \item $A \proves \phi \land \psi$.
  \item $A \proves \phi$ and $A \proves \psi$.
  \end{enumerate}

\item Assume $\phi$ is a sentence. Show (without using the completeness theorem) that $\proves (\exists x) \neg \phi \leftrightarrow \neg (\forall x) \phi$. \emph{Hint: do not hesitate to use the deduction theorem, the ``proof by contradiction'' lemma, as well as the introduction and elimination rules for quantifiers.}  
\item Let $\sigma$ be a signature with at least one constant symbol. Assume $A$ is a set of sentences in the language of $\sigma$. Solve the following two problems \emph{without} using the completeness theorem.

  \begin{enumerate}

  \item Let $X$ be the set of all closed terms (i.e.\ terms without free variables) in the language of $\sigma$. Define a relation $\sim$ on $X$ by $t \sim s$ if $A \proves t = s$. Prove that $\sim$ is an equivalence relation.
  \item Let $f$ be a function symbol of arity $n$ and let $t_1, \ldots, t_n, s_1, \ldots, s_n$ be closed terms. Show that if $A \proves t_i = s_i$ for all $i \le n$, then $A \proves f (t_1, \ldots, t_n) = f (s_1, \ldots, s_n)$.
  \end{enumerate}
\item For a fixed signature $\sigma$ and $A$, $B$ sets of sentences in the language of $\sigma$, we write $A \proves B$  if $A \proves \phi$ for every $\phi \in B$.

  \begin{enumerate}
  \item Prove without using the completeness theorem that if $A \proves B$ and $B \proves C$, then $A \proves C$. 
  \item Use the completeness theorem to give another proof that if $A \proves B$ and $B \proves C$, then $A \proves C$.
  \end{enumerate}
  
\end{enumerate}

\bibliographystyle{amsalpha}
\bibliography{notes}

\end{document}
