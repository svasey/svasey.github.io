\documentclass{amsart}
\usepackage{amssymb}
\usepackage{hyperref}
\usepackage{graphicx}

\theoremstyle{definition}
\newtheorem{mydef}{Definition}[section]
\newtheorem{lem}[mydef]{Lemma}
\newtheorem{thm}[mydef]{Theorem}
\newtheorem{cor}[mydef]{Corollary}
\newtheorem{claim}[mydef]{Claim}
\newtheorem{question}[mydef]{Question}
\newtheorem{hypothesis}[mydef]{Hypothesis}
\newtheorem{prop}[mydef]{Proposition}
\newtheorem{defin}[mydef]{Definition}
\newtheorem{example}[mydef]{Example}
\newtheorem{remark}[mydef]{Remark}
\newtheorem{notation}[mydef]{Notation}
\newtheorem{fact}[mydef]{Fact}

\newcommand{\bA}{\mathbf{A}}
\newcommand{\bB}{\mathbf{B}}
\newcommand{\bC}{\mathbf{C}}
\newcommand{\bD}{\mathbf{D}}

\newcommand{\seq}[1]{\langle #1 \rangle}

\title[Math 141a, Fall 2018: assignment 2]{Math 141a - Mathematical logic I, Fall 2018 \\ Assignment 2}

%% Include only sections, not subsections, in the table of content.
\setcounter{tocdepth}{1}

%% \author{Sebastien Vasey}
%% \email{sebv@cmu.edu}
%% \address{Department of Mathematical Sciences, Carnegie Mellon University, Pittsburgh, Pennsylvania, USA}
\date{\today}

\begin{document}

%% No indentation at the start of each paragraph
%\parindent 0pt

\maketitle

\textbf{Due Friday, September 21 at the beginning of class} (please submit your assignment on Canvas). Make sure to include your full name \emph{and the list of your collaborators} (if any) with your assignment. You may discuss problems with others, but you may \emph{not} keep a written record of your discussions. Please refer to the syllabus for details.

With regards to answering these problems, imagine that you are writing an answer to teach someone else in the class how to do the problem. In particular, you must give a complete outline for how you arrived at your answer. It is not sufficient to simply state a number or formula without providing the steps and reasoning that you used to produce the answer.

This assignment relies on the following facts and definitions (which have all been discussed in class):

\begin{itemize}
\item \textbf{Theorem.} (8.1,8.2 in Poizat) For any two well-orderings $\bA = (A, <)$ and $\bB = (B, <)$, either $\bA$ is isomorphic to an initial segment of $\bB$ or $\bB$ is isomorphic to an initial segment of $\bA$. In either case, the isomorphism is unique.

\item A set $X$ is \emph{transitive} if $x \subseteq X$ for any $x \in X$.
\item An \emph{ordinal} is a transitive set $\alpha$ such that $(\alpha, \in)$ is a well-ordering. For ordinals $\alpha$ and $\beta$, we write $\alpha < \beta$ instead of $\alpha \in \beta$. We write $\omega$ for the first infinite ordinal and identify the finite ordinals with the natural numbers (so $0 = \emptyset$, $1 = \{0\}$, etc.).
\item \textbf{Theorem.} (8.3 in Poizat) If two ordinals are isomorphic (as well-orderings), then they are equal.
\item \textbf{Theorem.} (8.4 in Poizat) Any well-ordering is (uniquely) isomorphic to a unique ordinal.
\item \textbf{Theorem.} (p.~163 in Poizat) Any non-empty collection of ordinals has a minimal element.

\item For an ordinal $\alpha$, $\alpha + 1 = \alpha \cup \{\alpha\}$. This is the minimal ordinal strictly above $\alpha$.
\item For a sequence of ordinals $\seq{\alpha_i : i \in I}$, $\sup_{i \in I} \alpha_i = \bigcup_{i \in I} \alpha_i$. This is the minimal ordinal $\alpha$ such that $\alpha_i \le \alpha$ for all $i \in I$.
\item An ordinal $\alpha$ is a \emph{successor} if $\alpha = \beta + 1$ for some ordinal $\beta$. It is \emph{limit} if it is not zero and not a successor. Note that if $\alpha$ is limit, then $\alpha = \sup_{\beta < \alpha} \beta$.
\item The \emph{principle of transfinite induction for ordinals} says that if $P (x)$ is a property of ordinals and for all ordinals $\alpha$, $P(0)$, $P (\alpha)$ implies $P (\alpha + 1)$, and $P (\beta)$ for all $\beta < \alpha$ implies $P (\alpha)$, then $P (\alpha)$ holds for all $\alpha$.
\item For ordinals $\alpha$ and $\beta$, $\alpha + \beta$ is the unique ordinal isomorphic to $(\alpha, \in) + (\beta, \in)$ (see assignment 1). Similarly, $\alpha \cdot \beta$ is the unique ordinal isomorphic to $(\alpha, \in) \times (\beta, \in)$.
\item Equivalently, one can inductively define $\alpha + 0 = \alpha$, $\alpha + (\beta + 1) = (\alpha + \beta) + 1$, and $\alpha + \beta = \sup_{\gamma < \beta} \alpha + \gamma$ for $\beta$ limit. One can then define ordinal multiplication: $\alpha \cdot 0 = 0$, $\alpha \cdot (\beta + 1) = \alpha \cdot \beta + \alpha$, and $\alpha \cdot \beta = \sup_{\gamma < \beta} \alpha \cdot \gamma$ for $\beta$ limit. You may take it for granted that these two definitions of addition/multiplication are equivalent.
\item Define $\alpha^\beta$ by induction on $\beta$ as follows: $\alpha^0 = 1$, $\alpha^{\beta + 1} = \alpha^\beta \cdot \alpha$, and $\alpha^\beta = \sup_{\gamma < \beta} \alpha^\gamma$ for $\beta$ limit.
\item \textbf{Theorem.} (8.6 in Poizat) Any set can be well-ordered.
\item The \emph{cardinality} of a set $X$, denoted $|X|$, is the minimal ordinal isomorphic to a well-ordering of $X$. An ordinal $\alpha$ is a \emph{cardinal} if $|\alpha| = \alpha$.
\item \textbf{Theorem.} (8.8 in Poizat) $|X| < |\mathcal{P} (X)|$ for any set $X$.
\item For a cardinal $\kappa$, $\kappa^+$ is the minimal cardinal strictly greater than $\kappa$. Define inductively $\aleph_0 = \omega$, $\aleph_{\alpha + 1} = \left(\aleph_\alpha\right)^+$, and $\aleph_{\alpha} = \sup_{\beta < \alpha} \aleph_\beta$ for $\alpha$ a limit ordinal.
\item \textbf{Theorem.} (seen in class) For any infinite cardinal $\lambda$, there exists an ordinal $\alpha$ such that $\lambda = \aleph_\alpha$.
\end{itemize}
\begin{enumerate}
\item \begin{enumerate}
\item Show that if $\beta$ is an ordinal and $\alpha \in \beta$, then $\alpha$ is an ordinal.
\item Show that if $\beta$ is an ordinal and $A$ is an initial segment of $(\beta, \in)$, then $A$ is an ordinal.
\item Show that for ordinals $\alpha$ and $\beta$, $\alpha \in \beta$ if and only if $\alpha \subsetneq \beta$.
\item Show that if $\alpha$ and $\beta$ are ordinals, then either $\alpha \le \beta$ or $\beta < \alpha$.
\end{enumerate}
\item \begin{enumerate}
\item Prove that any ordinal $\alpha$ can be written uniquely as $\alpha = \alpha_0 + n$, where $n < \omega$ and $\alpha_0$ is limit or zero. \emph{Hint: use transfinite induction on $\alpha$.}
\item Prove that for any two ordinals $\alpha \le \beta$, there exists a unique ordinal $\delta$ such that $\beta = \alpha + \delta$ \emph{Note: this shows that there is a kind of subtraction. Hint: proceed by transfinite induction on $\beta$.}
\item Prove that for any ordinal $\alpha$ and any ordinal $\beta > 0$, there exists unique ordinals $\gamma$ and $\delta$ such that $\delta < \beta$ and $\alpha = \beta \cdot \gamma + \delta$ \emph{Note: this shows there is a kind of division. Hint: take $\gamma$ minimal such that $\beta \cdot (\gamma + 1) > \alpha$, then use the previous part to find $\delta$.} 
\end{enumerate}
\item Let $\gamma$ be a nonzero ordinal. Show that for any ordinal $\alpha$, there exists $n < \omega$ and ordinals $\alpha_0 > \alpha_1 > \ldots > \alpha_{n - 1}$, $c_0, \ldots, c_{n - 1}$ such that $0 < c_i < \gamma$ for all $i < n$ and $\alpha = \gamma^{\alpha_0} c_0 + \gamma^{\alpha_1} c_1 + \ldots + \gamma^{\alpha_{n - 1}}c_{n - 1}$ (with the convention that when $n = 0$, the empty sum is zero). \emph{Note: This representation is called the base $\gamma$ normal form of $\alpha$. It is tedious but possible to show that this representation is also unique. Hint: proceed by induction on $\alpha$: to start, take $\alpha_0$ to be minimal such that $\gamma^{\alpha_0 + 1} > \alpha$ and use a previous exercise to ``divide'' by $\gamma^{\alpha_0}$.}
\item Write the following expression in base $\omega$ normal form (no full proof needed, you only need to justify briefly). For example, $\omega + \omega = \omega^1 \cdot 2$ in base $\omega$ normal form.
  \begin{enumerate}
  \item $3 + \omega + 5 + \omega$.
  \item $\omega^2 \cdot 3 + 5 \cdot \omega^3$.
  \item $(\omega \cdot 3) \cdot (\omega \cdot 5)$.
  \item $2^{32 \cdot (1 + \omega) + 2^\omega}$ (\emph{Warning: exponentiation here is the ordinal exponentiation defined above.})
  \end{enumerate}
\item Show that for sets $X$ and $Y$, $|X| = |Y|$ if and only if there is a bijection from $X$ onto $Y$.
\item Show that for sets $X$ and $Y$, the following are equivalent:

  \begin{enumerate}
  \item $|X| \le |Y|$.
  \item There is a surjection from $Y$ onto $X$.
  \item There is an injection from $X$ to $Y$.
  \end{enumerate}

  \emph{Hint: to go from the second to the third, draw a picture to understand the situation and use the well-ordering of $Y$ to ``choose'' where to send each element.}
\item (Extra credit) Show that there exists an infinite cardinal $\lambda$ such that $\lambda = \aleph_\lambda$.
\end{enumerate}

\bibliographystyle{amsalpha}
\bibliography{notes}

\end{document}
