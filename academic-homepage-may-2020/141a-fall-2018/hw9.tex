\documentclass{amsart}
\usepackage[utf8]{inputenc}
\usepackage{amssymb}
\usepackage{hyperref}
\usepackage{graphicx}

\theoremstyle{definition}
\newtheorem{mydef}{Definition}[section]
\newtheorem{lem}[mydef]{Lemma}
\newtheorem{thm}[mydef]{Theorem}
\newtheorem{cor}[mydef]{Corollary}
\newtheorem{claim}[mydef]{Claim}
\newtheorem{question}[mydef]{Question}
\newtheorem{hypothesis}[mydef]{Hypothesis}
\newtheorem{prop}[mydef]{Proposition}
\newtheorem{defin}[mydef]{Definition}
\newtheorem{example}[mydef]{Example}
\newtheorem{remark}[mydef]{Remark}
\newtheorem{notation}[mydef]{Notation}
\newtheorem{fact}[mydef]{Fact}

\newcommand{\bA}{\mathbf{A}}
\newcommand{\bB}{\mathbf{B}}
\newcommand{\bC}{\mathbf{C}}
\newcommand{\bD}{\mathbf{D}}

\newcommand{\ba}{\bar{a}}
\newcommand{\bb}{\bar{b}}

\newcommand{\bx}{\bar{x}}
\newcommand{\by}{\bar{y}}

\newcommand{\seq}[1]{\langle #1 \rangle}
\newcommand{\fct}[2]{{}^{#1}{#2}}
\newcommand{\rest}{\upharpoonright}

\newcommand{\proves}{\vdash}
\newcommand{\imp}{\rightarrow}
\newcommand{\impp}{\leftrightarrow}
\newcommand{\PA}{\operatorname{PA}}

\title[Math 141a, Fall 2018: assignment 9]{Math 141a - Mathematical logic I, Fall 2018 \\ Assignment 9}

%% Include only sections, not subsections, in the table of content.
\setcounter{tocdepth}{1}

%% \author{Sebastien Vasey}
%% \email{sebv@cmu.edu}
%% \address{Department of Mathematical Sciences, Carnegie Mellon University, Pittsburgh, Pennsylvania, USA}
\date{\today}

\begin{document}

%% No indentation at the start of each paragraph
%\parindent 0pt

\maketitle

\textbf{Due Friday, November 9 at the beginning of class} (please submit your assignment on Canvas). Make sure to include your full name \emph{and the list of your collaborators} (if any) with your assignment. You may discuss problems with others, but you may \emph{not} keep a written record of your discussions. Please refer to the syllabus for details.

With regards to answering these problems, imagine that you are writing an answer to teach someone else in the class how to do the problem. In particular, you must give a complete outline for how you arrived at your answer. It is not sufficient to simply state a number or formula without providing the steps and reasoning that you used to produce the answer.

\begin{enumerate}
\item Prove that for any signature $\sigma$ and any cardinal $\lambda \ge |\sigma| + \aleph_0$, there are at most $2^{\lambda}$-many non-isomorphic $\sigma$-structures of cardinality $\lambda$. In other words, if $(M_i)_{i < \left(2^\lambda\right)^+}$ are $\sigma$-structures of cardinality $\lambda$, there must exist $i < j < \left(2^{\lambda}\right)^+$ such that $M_i \cong M_j$. \emph{Hint: restrict yourself to a fixed universe.}
\item Assume that $A$ is a set of sentences in the language of a signature $\sigma$. For a cardinal $\lambda$, we say that $A$ is \emph{categorical} in $\lambda$ if $A$ has exactly one model of cardinality $\lambda$ up to isomorphism. That is, $A$ has a model $M$ of cardinality $\lambda$, and any other model $N$ of $A$ of cardinality $\lambda$ is isomorphic to $M$.

  Prove that if $A$ has no finite models and $A$ is categorical in some cardinal $\lambda$ such that $\lambda \ge |\sigma| + \aleph_0$, then $A$ is complete.
\item Let $N$ be a model of $\PA$. We call $a \in N$ \emph{superdivisible} if it is not zero and for every standard non-zero natural number $n$, $n$ divides $a$. More precisely, for every nonzero $n < \omega$,

  $$
  N\models a \neq 0 \land (\exists x)( a = x \cdot (\underbrace{1 + 1 + \ldots + 1}_{n \text{ times}}))
  $$

  We also fix more notation: for $a \in N$ nonstandard and $k$ a natural number, we write $(a + k)^N$ for $(a + 0)^N$ if $k = 0$ or $(a + \underbrace{1 + 1 + \ldots + 1}_{k \text{ times}})^N$ if $k > 0$. For $k$ a strictly negative integer, we write $(a + k)^N$ for the unique $b \in N$ such that $(b + |k|)^N = a$.

  \emph{Note: for the problems below, you do not need to explain in details why a fact about the natural numbers follows from the axioms of $\PA$. Just make sure to convince yourself that it is true.}

  \begin{enumerate}
  \item Show that there exists a model of $\PA$ which has a superdivisible element.
  \item Let $N$ be a model of $\PA$ with superdivisible elements. Show that:

    \begin{enumerate}
    \item If $a$ and $b$ are superdivisible, then $a +^N b$ is superdivisible.
    \item If $a$ is superdivisible and $b$ is any nonzero element of $N$, then and $a \cdot^N b$ is superdivisible.
    \item If $a \in N$ is superdivisible, then $(a + k)^N$ is \emph{not} superdivisible for any nonzero integer $k$.
    \item If $a \in N$ is superdivisible and $n < \omega$ is not zero, then $\frac{a}{n}$ is superdivisible. More formally, there exists a unique superdivisible $b$ such that $N \models a = b \cdot (\underbrace{1 + 1 + \ldots + 1}_{n \text{ times}})$.
    \end{enumerate}
  \item Let $N$ be a model of $\PA$ with superdivisible elements. Let $X = \mathbb{N} \cup \{(a + k)^N \mid a \text{ superdivisible}, k \in \mathbb{Z}\}$. Show that $X$ is closed under addition and multiplication in $N$.
  \item Write a formula $\phi (x)$ in the language of $\PA$ such that $(\mathbb{N}, +, \cdot, 0, 1) \models \phi (n)$ if and only if $n$ is prime.
  \item Show that Goldbach's conjecture fails in the substructure $M$ of $N$ with universe $X$ (more precisely, the formula coding Goldbach's conjecture in the language of $\PA$ is false in $M$). \emph{(It is unclear whether $M$ is a model of $\PA$ -- you get nonstandard bonus points if you can figure it out.)}
  \end{enumerate} 

\end{enumerate}

\bibliographystyle{amsalpha}
\bibliography{notes}

\end{document}
