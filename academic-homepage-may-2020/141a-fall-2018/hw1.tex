\documentclass{amsart}
\usepackage{hyperref}
\usepackage{graphicx}

\theoremstyle{definition}
\newtheorem{mydef}{Definition}[section]
\newtheorem{lem}[mydef]{Lemma}
\newtheorem{thm}[mydef]{Theorem}
\newtheorem{cor}[mydef]{Corollary}
\newtheorem{claim}[mydef]{Claim}
\newtheorem{question}[mydef]{Question}
\newtheorem{hypothesis}[mydef]{Hypothesis}
\newtheorem{prop}[mydef]{Proposition}
\newtheorem{defin}[mydef]{Definition}
\newtheorem{example}[mydef]{Example}
\newtheorem{remark}[mydef]{Remark}
\newtheorem{notation}[mydef]{Notation}
\newtheorem{fact}[mydef]{Fact}

\newcommand{\bA}{\mathbf{A}}
\newcommand{\bB}{\mathbf{B}}
\newcommand{\bC}{\mathbf{C}}
\newcommand{\bD}{\mathbf{D}}

\title[Math 141a, Fall 2018: assignment 1]{Math 141a - Mathematical logic I, Fall 2018 \\ Assignment 1}

%% Include only sections, not subsections, in the table of content.
\setcounter{tocdepth}{1}

%% \author{Sebastien Vasey}
%% \email{sebv@cmu.edu}
%% \address{Department of Mathematical Sciences, Carnegie Mellon University, Pittsburgh, Pennsylvania, USA}
\date{\today}

\begin{document}

%% No indentation at the start of each paragraph
%\parindent 0pt

\maketitle

\textbf{Due Friday, September 14 at the beginning of class} (please submit your assignment on Canvas). Make sure to include your full name \emph{and the list of your collaborators} (if any) with your assignment. You may discuss problems with others, but you may \emph{not} keep a written record of your discussions. Please refer to the syllabus for details.

With regards to answering these problems, imagine that you are writing an answer to teach someone else in the class how to do the problem. In particular, you must give a complete outline for how you arrived at your answer. It is not sufficient to simply state a number or formula without providing the steps and reasoning that you used to produce the answer.

This assignment relies on the following definitions (most appear on p.~7-10 of Poizat and have or will soon be discussed in class):

\begin{itemize}
\item A \emph{chain} (also called a \emph{linear order}) is a pair $\mathbf{A} = (A, <)$ where $A$ is a set and $<$ is a binary relation on $A$. We require that $<$ is transitive, irreflexive, and total (either $a < b$ or $b < a$ whenever $a \neq b$ are in $A$). As usual, we write $a \le b$ for ``$a < b$ or $a = b$''.
\item A \emph{well-ordering} is a chain $\bA = (A, <)$ where every non-empty subset has a minimal element (i.e.\ for every non-empty $S \subseteq A$ there exists $a \in S$ such that $a \le b$ for any $b \in S$). 
  \item For $\mathbf{A}$ and $\mathbf{B}$ chains, a function $f: \mathbf{A} \to \mathbf{B}$ is \emph{order-preserving} (or \emph{increasing}) if $a < b$ implies $f (a) < f (b)$. We say that $f$ is an \emph{isomorphism} from $\mathbf{A}$ to $\mathbf{B}$ if $f$ is an order-preserving bijection. In this case, we write $f: \mathbf{A} \cong \mathbf{B}$. We say that $\mathbf{A}$ and $\mathbf{B}$ are \emph{isomorphic} (written $\mathbf{A} \cong \mathbf{B}$) if there exists an isomorphism from $\mathbf{A}$ to $\mathbf{B}$.
  \item The \emph{inverse} of a chain $\bA = (A, <)$ is the chain $\bA^- = (A, <^-)$, where $<^-$ is the ``inverse'' order: $a <^- b$ if and only if $b < a$.
  \item The \emph{sum} of two chains $\mathbf{A}_1 = (A_1, <_1), \mathbf{A}_2 = (A_2, <_2)$, written $\mathbf{A}_1 + \mathbf{A}_2$, is the pair $((A_1 \times \{1\}) \cup (A_2 \times \{2\}), <)$, where $(a, \ell_1) < (b, \ell_2)$ is true exactly when one of the following conditions hold:
    \begin{itemize}
    \item $\ell_1 < \ell_2$.
    \item $\ell_1 = \ell_2$ and $a <_{\ell_1} b$.
    \end{itemize}

    In other words, $\mathbf{A}_1 + \mathbf{A}_2$ puts all the elements of $\mathbf{A}_2$ after the elements of $\mathbf{A}_1$.
  \item The \emph{product} of two chains $\mathbf{A}_1 = (A_1, <_1), \mathbf{A}_2 = (A_2, <_2)$, written $\mathbf{A}_1 \times \mathbf{A}_2$ is $(A_1 \times A_2, <)$, where $(a_1, a_2) < (b_1, b_2)$ is true exactly when one of the following conditions hold:

    \begin{itemize}
    \item $a_2 <_2 b_2$.
    \item $a_2 = b_2$ and $a_1 <_1 b_1$.
    \end{itemize}

    (This would be the dictionary order of a language written right to left, such as arabic).
\end{itemize}

\begin{enumerate}
\item (Extra credit: 12.5\%) \begin{enumerate}
\item Please fill in the survey at: \\
  \url{http://math.harvard.edu/~sebv/141a-fall-2018/questionnaire.odt}. Submit it separately on Canvas.
\item I like to know my students as human beings, so I would like to have a short one on one 5-10 min chat with you during the first few weeks of the semester, just so that I can know your face, name, and a little bit about your background. Don't be afraid, we're not going to talk math (unless you really want to!). You don't need to prepare anything for the meeting.

  Please send me a short email at \url{sebv@math.harvard.edu} with subject ``141a short meeting'' and ask e.g.\ ``is 1pm next Monday okay?''. I will either reply yes or give you another time. The meeting will take place in my office, SC 321H. \emph{Note that I am out of town Sep. 11 and Sep. 12!}
\end{enumerate}
\item \begin{enumerate}
\item Show that any order-preserving function is injective.
\item Show that if $f$ is an isomorphism, then its inverse $f^{-1}$ is also an isomorphism.
\end{enumerate}
\item Show that the sum of two chains is a chain. Also show that the product of two chains is a chain.
\item Show that the sum of two well-orderings is a well-ordering. Also show that the product of two well-orderings is a well-ordering.
\item Show that any two chains of the same finite cardinality are isomorphic.
\item Show that $\mathbb{Z}$ (with its natural ordering) is isomorphic to its inverse, but not isomorphic to $\mathbb{Q}$.
\item Give an example of a chain which is not isomorphic to its inverse (and prove it).
\item Show that for any chains $\bA$, $\bB$, and $\bC$, we have $\bA \times (\bB + \bC) \cong (\bA \times \bB) + (\bA \times \bC)$.
\item Let $\mathbf{1} = (\{0\}, <)$ be the one element chain and let $\mathbb{N}$ be the natural numbers with their usual ordering. Show that $\mathbf{1} + \mathbb{N} \cong \mathbb{N}$ but $\mathbb{N} \not \cong \mathbb{N} + \mathbf{1}$.
\end{enumerate}

\bibliographystyle{amsalpha}
\bibliography{notes}

\end{document}
