\documentclass{amsart}
\usepackage[utf8]{inputenc}
\usepackage{amssymb}
\usepackage{hyperref}
\usepackage{graphicx}

\theoremstyle{definition}
\newtheorem{mydef}{Definition}[section]
\newtheorem{lem}[mydef]{Lemma}
\newtheorem{thm}[mydef]{Theorem}
\newtheorem{cor}[mydef]{Corollary}
\newtheorem{claim}[mydef]{Claim}
\newtheorem{question}[mydef]{Question}
\newtheorem{hypothesis}[mydef]{Hypothesis}
\newtheorem{prop}[mydef]{Proposition}
\newtheorem{defin}[mydef]{Definition}
\newtheorem{example}[mydef]{Example}
\newtheorem{remark}[mydef]{Remark}
\newtheorem{notation}[mydef]{Notation}
\newtheorem{fact}[mydef]{Fact}

\newcommand{\bA}{\mathbf{A}}
\newcommand{\bB}{\mathbf{B}}
\newcommand{\bC}{\mathbf{C}}
\newcommand{\bD}{\mathbf{D}}

\newcommand{\ba}{\bar{a}}
\newcommand{\bb}{\bar{b}}

\newcommand{\bx}{\bar{x}}
\newcommand{\by}{\bar{y}}

\newcommand{\seq}[1]{\langle #1 \rangle}
\newcommand{\fct}[2]{{}^{#1}{#2}}
\newcommand{\rest}{\upharpoonright}

\title[Math 141a, Fall 2018: assignment 6]{Math 141a - Mathematical logic I, Fall 2018 \\ Assignment 6}

%% Include only sections, not subsections, in the table of content.
\setcounter{tocdepth}{1}

%% \author{Sebastien Vasey}
%% \email{sebv@cmu.edu}
%% \address{Department of Mathematical Sciences, Carnegie Mellon University, Pittsburgh, Pennsylvania, USA}
\date{\today}

\begin{document}

%% No indentation at the start of each paragraph
%\parindent 0pt

\maketitle

\textbf{Due Monday, October 22 at the beginning of class} (please submit your assignment on Canvas). Make sure to include your full name \emph{and the list of your collaborators} (if any) with your assignment. You may discuss problems with others, but you may \emph{not} keep a written record of your discussions. Please refer to the syllabus for details.

With regards to answering these problems, imagine that you are writing an answer to teach someone else in the class how to do the problem. In particular, you must give a complete outline for how you arrived at your answer. It is not sufficient to simply state a number or formula without providing the steps and reasoning that you used to produce the answer.

\begin{enumerate}
\item Let $M$ and $N$ be $\sigma$-structures. Assume that $s$ is an isomorphism from $M$ to $N$, let $a_1, a_2, \ldots, a_n$ be members of in the universe of $M$, and let $\phi (x_1, \ldots, x_n)$ be a formula. Prove that $M \models \phi (a_1, \ldots, a_n)$ if and only if $N \models \phi (s (a_1), \ldots, s (a_n))$.
\item For a fixed $\sigma$-structure $M$, and a number $n < \omega$, a set $X$ of $n$-tuples in the universe of $M$ is called \emph{definable} if there exists a formula $\phi (x_1, \ldots, x_n)$ (in the language of $\sigma$) such that for any $n$-tuple $\ba$ from $M$, $M \models \phi (\ba)$ if and only if $\ba \in X$.

  \begin{enumerate}
  \item Explain why any infinite structure of cardinality strictly larger than $|\sigma|$  will have non-definable subsets.
  \item Assume $M$ is a $\sigma$-structure with universe $E$ and let $n < \omega$. Show that $\emptyset$ and $E^n$ are always definable.
  \item Show that definable sets are closed under complements, finite unions, and finite intersections. 
  \item Consider the structure $M = (\mathbb{R}, +, \cdot)$. Show that the sets $\{0\}$ and $\{1\}$ are definable in $M$. Further show that the set $\{(a, b) \in \mathbb{R} \times \mathbb{R} \mid a < b\}$ is also definable.
  \item Show that $\mathbb{N}$ is definable in $(\mathbb{Z}, +, \cdot)$, but not in $(\mathbb{Q}, <)$. \emph{Hint: for the first part, you may want to use the following theorem from number theory: any natural number can be written as the sum of four squares.}
  \end{enumerate}

\item \begin{enumerate}

\item Prove that the following are equivalent for a chain $(C, <)$:
  \begin{enumerate}
  \item $C$ is \emph{not} a well-ordering.
  \item There exists a sequence $(a_n)_{n < \omega}$ such that $a_{n + 1} < a_n$ for all $n < \omega$.
  \end{enumerate}
\item Assume $C$ is an infinite well-ordering. Show that there exists a chain $D$ which is elementary equivalent to $C$ but is not a well-ordering. \emph{Hint: add constant symbols and use the compactness theorem.}
\end{enumerate}
\item Assume $P$ is a (possibly infinite!) set of primes. Show that there exists a structure $M$ such that $M$ is elementarily equivalent to $(\mathbb{N}, +, \cdot, 0, 1)$ but there is $a$ in the universe of $M$ that is divisible exactly by the primes in $P$. That is, for any prime $p$, $p \in P$ if and only if

  $$
  M \models (\exists y)( a = (\underbrace{1 + 1 + \ldots + 1}_{p \text{ times}}) y)
  $$

  \emph{Hint: use the compactness theorem.}
\item Let $F$ be a filter on $\omega$. Let $(a_n)_{n < \omega}$ be a sequence of real numbers and let $a$ be a real number. We say that $(a_n)_{n < \omega}$ \emph{$F$-converges to $a$} if for any $\epsilon > 0$, $\{n < \omega \mid |a_n - a| < \epsilon\} \in F$. We call $a$ an \emph{$F$-limit} of $(a_n)_{n < \omega}$.

  \begin{enumerate}
  \item Show that if $(a_n)_{n < \omega}$ $F$-converges to both $a$ and $b$, then $a = b$. \emph{Note: this shows that the $F$-limit is unique if it exists.}
  \item Describe the $F$-limit if $F$ is the Fréchet filter. What if $F$ is a principal ultrafilter?
  \item Assume now that $U$ is an ultrafilter. Show that any bounded sequence $U$-converges.
  \end{enumerate}
\end{enumerate}

\bibliographystyle{amsalpha}
\bibliography{notes}

\end{document}
