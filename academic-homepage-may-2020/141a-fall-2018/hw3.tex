\documentclass{amsart}
\usepackage{amssymb}
\usepackage{hyperref}
\usepackage{graphicx}

\theoremstyle{definition}
\newtheorem{mydef}{Definition}[section]
\newtheorem{lem}[mydef]{Lemma}
\newtheorem{thm}[mydef]{Theorem}
\newtheorem{cor}[mydef]{Corollary}
\newtheorem{claim}[mydef]{Claim}
\newtheorem{question}[mydef]{Question}
\newtheorem{hypothesis}[mydef]{Hypothesis}
\newtheorem{prop}[mydef]{Proposition}
\newtheorem{defin}[mydef]{Definition}
\newtheorem{example}[mydef]{Example}
\newtheorem{remark}[mydef]{Remark}
\newtheorem{notation}[mydef]{Notation}
\newtheorem{fact}[mydef]{Fact}

\newcommand{\bA}{\mathbf{A}}
\newcommand{\bB}{\mathbf{B}}
\newcommand{\bC}{\mathbf{C}}
\newcommand{\bD}{\mathbf{D}}

\newcommand{\ba}{\bar{a}}
\newcommand{\bb}{\bar{b}}

\newcommand{\seq}[1]{\langle #1 \rangle}
\newcommand{\fct}[2]{{}^{#1}{#2}}
\newcommand{\rest}{\upharpoonright}

\title[Math 141a, Fall 2018: assignment 3]{Math 141a - Mathematical logic I, Fall 2018 \\ Assignment 3}

%% Include only sections, not subsections, in the table of content.
\setcounter{tocdepth}{1}

%% \author{Sebastien Vasey}
%% \email{sebv@cmu.edu}
%% \address{Department of Mathematical Sciences, Carnegie Mellon University, Pittsburgh, Pennsylvania, USA}
\date{\today}

\begin{document}

%% No indentation at the start of each paragraph
%\parindent 0pt

\maketitle

\textbf{Due Friday, September 28 at the beginning of class} (please submit your assignment on Canvas). Make sure to include your full name \emph{and the list of your collaborators} (if any) with your assignment. You may discuss problems with others, but you may \emph{not} keep a written record of your discussions. Please refer to the syllabus for details.

With regards to answering these problems, imagine that you are writing an answer to teach someone else in the class how to do the problem. In particular, you must give a complete outline for how you arrived at your answer. It is not sufficient to simply state a number or formula without providing the steps and reasoning that you used to produce the answer.

This assignment relies on the following facts and definitions (which have all been discussed in class):

\begin{itemize}
\item A set $A$ is \emph{countable} if $|A| \le \aleph_0$. We call $A$ \emph{finite} if $|A| < \aleph_0$ and \emph{infinite} otherwise. We say that $A$ is \emph{countably infinite}, or \emph{denumerable}, if $|A| = \aleph_0$.
\item For sets $A$ and $B$, $\fct{A}{B}$ denotes the set of functions from $A$ to $B$. More precisely, $\fct{A}{B}$ is the set of all relations $R \subseteq A \times B$ such that for any $a \in A$, there exists a unique $b \in B$ with $(a, b) \in R$. Note in particular that $\fct{A}{B} \subseteq \fct{A}{B'}$ for $B \subseteq B'$.
\item \emph{Cardinal addition, multiplication, and exponentiation} (which are different from the corresponding ordinal operations) are defined as follows. For $\lambda$ and $\mu$ cardinals:
  \begin{itemize}
  \item $\lambda + \mu = |(\lambda \times \{1\}) \cup (\mu \times \{2\})|$.
  \item $\lambda \cdot \mu = |\lambda \times \mu|$.
  \item $\lambda^\mu = |\fct{\mu}{\lambda}|$.
  \end{itemize}
\item \textbf{Theorem.} (see 8.9 and the few pages after in Poizat) If $\lambda$ and $\mu$ are both infinite cardinals, then $\lambda + \mu = \lambda \cdot \mu = \max (\lambda, \mu)$.
\item A chain $\bA = (A, <)$ is \emph{dense} if for any $x < y$ in $A$, there exists $z \in A$ such that $x < z < y$. An \emph{endpoint} of a chain $\bA$ is a minimum or a maximum.
\item \textbf{Theorem.} (a special case of 1.14 in Poizat) Any two non-empty countable dense chains without endpoints are isomorphic.
\item We will also rely on some of the other definitions given in Chapter 1 of Poizat, such as that of a local isomorphism and of an $\alpha$-isomorphism. Two relations $R$ and $R'$ are \emph{elementarily equivalent} if there is an $\omega$-isomorphism from $R$ to $R'$.
\item For two relations $R$ and $R'$, we say that a tuple $\ba = (a_1, \ldots, a_k)$ from the universe of $R$ is \emph{$\alpha$-equivalent} to a tuple $\bb = (b_1, b_2, \ldots, b_k)$ from the universe of $R'$ (written $(\ba, R) \sim_\alpha (\bb, R')$) if there is an $\alpha$-isomorphism from $R$ to $R'$ taking $\ba$ to $\bb$. That is, sending $a_i$ to $b_i$ for each $i \le k$ defines an $\alpha$-isomorphism. See the end of p.~3 in Poizat.

\end{itemize}

\begin{enumerate}
\item
  \begin{enumerate}
  \item Fix an infinite cardinal $\mu$. Give an example of an infinite cardinal $\lambda$ such that $\lambda^{\mu} = \lambda$.
  \item Let $\seq{A_i : i \in I}$ be a non-empty sequence of sets. Show that:

    $$
    \left| \bigcup_{i \in I} A_i \right| \le |I| \cdot \sup_{i \in I} |A_i| 
    $$
  \item Let $\lambda$ be an infinite cardinal. Let $A \subseteq \lambda^+$ be such that $|A| \le \lambda$. Show that there exists an ordinal $\alpha < \lambda^+$ such that $A \subseteq \alpha$. \emph{Hint: suppose not, and write $\lambda^+$ as a union of $\lambda$-many sets of cardinality $\lambda$.}
  \item Let $\mu \le \lambda$ be infinite cardinals. Show that $\fct{\mu}{\left(\lambda^+\right)} = \bigcup_{\alpha < \lambda^+} \fct{\mu}{\alpha}$.

  \item Let $\lambda$ and $\mu$ be infinite cardinals. Assume that $\lambda^\mu = \lambda$. Prove that $\left(\lambda^+\right)^\mu = \lambda^+$.
  \end{enumerate}
\item (Exercise 1.1 in Poizat) Let $R$ and $R'$ be $m$-ary relations with universes $E$ and $E'$ respectively. Show that if $R$ and $R'$ are elementarily equivalent, then for every $k < \omega$, every $p < \omega$, and every $k$-tuple $\ba \in E^k$, there is a $k$-tuple $\bb \in (E')^k$ such that $(\ba, R) \sim_p (\bb, R')$.
\item How many countable dense chains are there, up to isomorphism?
\item Show that the chains $(\mathbb{R}, <)$ and $(\mathbb{R} \backslash \{0\}, <)$ are \emph{not} isomorphic.
\item A \emph{graph} is a pair $(V, E)$, where $V$ is a set and $E$ is a binary relation on $A$ that is symmetric and irreflexive (we think of $V$ as a set of \emph{vertices} and $E$ as a set of \emph{edges} connecting the vertices). A graph is called \emph{generic} if for any finite set $V_0 \subseteq V$ and any $S \subseteq V_0$, there exists $v \in V \backslash V_0$ such that for all $w \in V_0$, $E (v, w)$ if and only if $w \in S$. In other words, $v$ is connected to all the vertices in $S$ and to none of the vertices in $V_0 \backslash S$.

  Prove that any two countable generic graphs are isomorphic.

  \emph{Note: there are several ways to see that generic graphs exist. One construction is to take the vertex set to be $\omega$, and for each $i < j$, toss a coin to decide whether or not there is an edge between $i$ and $j$ (with probability one, we then obtain a generic graph). We will see another construction later in this class. The unique countable generic graph is often called the random graph, or the Rado graph.}

\item (Extra credit) 

  Let $S$ be a non-empty set (think of it as the set of ``states'' of the physical universe). Show that there is a function $\sigma$ (the ``strategy'') that:
  \begin{itemize}
  \item Takes as input a function $g: (-\infty, t) \to S$, for some real $t$ (you can think of $g$ as giving the ``past'' state of the universe at any time $s < t$), and outputs a state $\sigma (g) \in S$.
  \item ``Predicts'' the present from the past correctly almost always, in the sense that for any function $f: \mathbb{R} \to S$, for all but countably-many $t \in \mathbb{R}$, $f (t) = \sigma (f \rest (-\infty, t))$.
  \end{itemize}

  \emph{Hint: Well-order the set of functions from $\mathbb{R}$ to $S$ and have $\sigma$ pick out the value given by the minimal one agreeing with the past. To prove that $\sigma$ works, prove that the set of points where $\sigma$ is wrong has to be well-ordered by the usual ordering on the reals. Conclude by showing (using that the rationals are dense in the reals) that any subset of reals that is well-ordered by the usual ordering on the reals must be countable.}
\end{enumerate}

\bibliographystyle{amsalpha}
\bibliography{notes}

\end{document}
