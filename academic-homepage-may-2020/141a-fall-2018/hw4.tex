\documentclass{amsart}
\usepackage{amssymb}
\usepackage{hyperref}
\usepackage{graphicx}

\theoremstyle{definition}
\newtheorem{mydef}{Definition}[section]
\newtheorem{lem}[mydef]{Lemma}
\newtheorem{thm}[mydef]{Theorem}
\newtheorem{cor}[mydef]{Corollary}
\newtheorem{claim}[mydef]{Claim}
\newtheorem{question}[mydef]{Question}
\newtheorem{hypothesis}[mydef]{Hypothesis}
\newtheorem{prop}[mydef]{Proposition}
\newtheorem{defin}[mydef]{Definition}
\newtheorem{example}[mydef]{Example}
\newtheorem{remark}[mydef]{Remark}
\newtheorem{notation}[mydef]{Notation}
\newtheorem{fact}[mydef]{Fact}

\newcommand{\bA}{\mathbf{A}}
\newcommand{\bB}{\mathbf{B}}
\newcommand{\bC}{\mathbf{C}}
\newcommand{\bD}{\mathbf{D}}

\newcommand{\ba}{\bar{a}}
\newcommand{\bb}{\bar{b}}

\newcommand{\bx}{\bar{x}}
\newcommand{\by}{\bar{y}}

\newcommand{\seq}[1]{\langle #1 \rangle}
\newcommand{\fct}[2]{{}^{#1}{#2}}
\newcommand{\rest}{\upharpoonright}

\title[Math 141a, Fall 2018: assignment 4]{Math 141a - Mathematical logic I, Fall 2018 \\ Assignment 4}

%% Include only sections, not subsections, in the table of content.
\setcounter{tocdepth}{1}

%% \author{Sebastien Vasey}
%% \email{sebv@cmu.edu}
%% \address{Department of Mathematical Sciences, Carnegie Mellon University, Pittsburgh, Pennsylvania, USA}
\date{\today}

\begin{document}

%% No indentation at the start of each paragraph
%\parindent 0pt

\maketitle

\textbf{Due Friday, October 5 at the beginning of class} (please submit your assignment on Canvas). Make sure to include your full name \emph{and the list of your collaborators} (if any) with your assignment. You may discuss problems with others, but you may \emph{not} keep a written record of your discussions. Please refer to the syllabus for details.

With regards to answering these problems, imagine that you are writing an answer to teach someone else in the class how to do the problem. In particular, you must give a complete outline for how you arrived at your answer. It is not sufficient to simply state a number or formula without providing the steps and reasoning that you used to produce the answer.

\begin{enumerate}
\item A formula $f$ is said to be in \emph{prenex form} if all its quantifiers occur at the beginning. That is, $f$ is of the form $Q_0 x_0 Q_1 x_1 \ldots Q_{n - 1} x_{n - 1} g$, where each $x_i$ is a variable, each $Q_i$ is either $\forall$ or $\exists$, and $g$ is a quantifier-free formula.

  Prove that any formula is equivalent to a formula in prenex form.

  You may follow the sketch of the proof given on p.~23 in Poizat, but give the details by proceeding by induction on the complexity of the formula; you can use without proofs the equivalences listed on p.~20 and p.~21 of Poizat. Also pay attention to the difference between ``almost equivalent'' and ``equivalent'' outlined on p.~22.
\item A quantifier-free formula $f$ is said to be in \emph{disjunctive form} if it is of the form $\bigvee_{i < n} f_i$, where each $f_i$ is of the form $\bigwedge_{j < n_i} f_{i, j}$, and $f_{i, j}$ is either atomic or the negation of an atomic formula.

  Prove that any quantifier-free formula is equivalent to a quantifier-free formula in disjunctive form.

\item Recall that a theory $T$ (in the language of an $m$-ary relation) is called \emph{complete} if $f \in T$ or $\neg f \in T$ for any sentence $f$ in the language of an $m$-ary relation.

  Assume $T$ is a theory. Prove that $T$ is complete if and only if any two models of $T$ are elementarily equivalent.

\item Recall that we call a consistent set of sentences $A$ \emph{complete} if its set $T_A$ of consequences is complete.

  \begin{enumerate}
  \item Give a set of axioms (in the language for a binary relation) whose models are exactly the  graphs (that is, the irreflexive and symmetric relations).
  \item Give a set of axioms (in the language for a binary relation) whose models are exactly the generic graphs (see the previous problem set).
  \item Prove that any set of axioms whose models are exactly the generic graphs must be complete.
  \end{enumerate}
\item For an $m$-ary relation $R$ with universe $E$ which is a restriction of a relation $R'$ with universe $E'$, we say that $R'$ is an \emph{elementary extension of $R$} (written $(E, R) \preceq (E', R')$, or just $R \preceq R'$), if for any tuple $\ba$ from $E$ and any formula $f (\bx)$, $R \models f (\ba)$ if and only if $R' \models f (\ba)$.

  \begin{enumerate}
  \item Assume that $R$ is an $m$-ary relation with universe $E$, $R'$ is a relation with universe $E'$, and $R$ is a restriction of $R'$. Prove that the following are equivalent:
    \begin{enumerate}
    \item For any tuple $\ba$ from $E$, $(\ba, R) \sim_{\omega} (\ba, R')$
    \item $R \preceq R'$.
    \end{enumerate}

    (note that the first item is the definition of elementary extension given on p.~4 of Poizat).
  \item Prove that $(\mathbb{N}, <) \not \preceq (\{-1\} \cup \mathbb{N}, <)$, but $(\mathbb{N}, <)$ and $(\{-1\} \cup \mathbb{N}, <)$ are elementarily equivalent.
  \item Prove that $(\mathbb{Q}, <) \preceq (\mathbb{R}, <)$ and $(\mathbb{Z}, <) \preceq (\mathbb{Z}, <) + (\mathbb{Z}, <)$. \emph{Hint: use Poizat's equivalent definition.}
  \end{enumerate}
\end{enumerate}

\bibliographystyle{amsalpha}
\bibliography{notes}

\end{document}
