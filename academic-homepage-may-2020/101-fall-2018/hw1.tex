\documentclass{amsart}
\usepackage{hyperref}
\usepackage{graphicx}

\theoremstyle{definition}
\newtheorem{mydef}{Definition}[section]
\newtheorem{lem}[mydef]{Lemma}
\newtheorem{thm}[mydef]{Theorem}
\newtheorem{cor}[mydef]{Corollary}
\newtheorem{claim}[mydef]{Claim}
\newtheorem{question}[mydef]{Question}
\newtheorem{hypothesis}[mydef]{Hypothesis}
\newtheorem{prop}[mydef]{Proposition}
\newtheorem{defin}[mydef]{Definition}
\newtheorem{example}[mydef]{Example}
\newtheorem{remark}[mydef]{Remark}
\newtheorem{notation}[mydef]{Notation}
\newtheorem{fact}[mydef]{Fact}

\title[Math 101, Fall 2018: assignment 1]{Math 101 - Sets, groups, and topology, Fall 2018 \\ Assignment 1}

%% Include only sections, not subsections, in the table of content.
\setcounter{tocdepth}{1}

%% \author{Sebastien Vasey}
%% \email{sebv@cmu.edu}
%% \address{Department of Mathematical Sciences, Carnegie Mellon University, Pittsburgh, Pennsylvania, USA}
\date{\today}

\begin{document}

%% No indentation at the start of each paragraph
%\parindent 0pt

\maketitle

\textbf{Due Monday, September 10 at the beginning of class} (please submit your assignment on Canvas). Make sure to include your full name \emph{and the list of your collaborators} (if any) with your assignment. You may discuss problems with others, but you may \emph{not} keep a written record of your discussions. Please refer to the syllabus for details.

\begin{enumerate}
\item (Extra credit: 25\%) \begin{enumerate}
\item Please fill in the survey at: \\
  \url{http://math.harvard.edu/~sebv/101-fall-2018/questionnaire.odt}. Submit it separately on Canvas.
\item I like to know my students as human beings, so I would like to have a short one on one 5-10 min chat with you during the first few weeks of the semester, just so that I can know your face, name, and a little bit about your background. Don't be afraid, we're not going to talk math (unless you really want to!). You don't need to prepare anything for the meeting.

  Please send me a short email at \url{sebv@math.harvard.edu} with subject ``101 short meeting'' and ask e.g.\ ``is 1pm next Monday okay?''. I will either reply yes or give you another time. The meeting will take place in my office, SC 321H. \emph{Note that I am out of town Sep. 11 and Sep.12!}
\end{enumerate}
\item Recall that $\pi$ is defined to be the ratio of a circle's circumference to its diameter (you may take it for granted that this ratio is always constant). A well known theorem of Archimedes says that the \emph{area} of a circle of radius $r$ is given by $\pi r^2$ (this is actually not that easy to prove). Show that $\pi \neq 4$. 

  You are allowed to use basic geometric facts and constructions, but not any results (except the ones just mentioned) about $\pi$ itself. If you draw a picture, explain precisely how it is drawn. If you want to claim a quantity is greater than another, make sure you justify why without relying too much on a (possibly imperfect) drawing. In general, try to justify each step as well as you can.
\item  What is (mathematically) wrong in the following ``proof'' that $0 = 1$?

Assume $x, y$ are real numbers such that $x = y$.

\begin{align*}
  x &= y \qquad &\text{(by assumption)} \\
  x - y &= 0 \qquad &\text{(Subtracting $y$ from both sides)} \\
  (x - y)(x - y)^{-1} &= 0 (x - y)^{-1} \qquad &\text{(Multiplying both sides by $(x - y)^{-1}$)} \\
  1 &= 0 (x - y)^{-1} \qquad &\text{($1 = a \cdot a^{-1}$ for any real number $a$)} \\
  1 &= 0 \qquad &\text{(Any number multiplied by zero is zero)}
\end{align*}

\item What is (mathematically) wrong in the following ``proof'' that all nonzero real numbers are positive?

  For $x$ a nonzero real number, we always have that $0 < x \cdot x$. Multiply both sides of the inequality by $x^{-1}$ to obtain $0 \cdot x^{-1} < x \cdot x \cdot x^{-1}$. Since any number multiplied by zero is zero, $0 < x \cdot x \cdot x^{-1}$. By properties of $x^{-1}$, $x \cdot x^{-1} = 1$, so $0 < x \cdot 1$. Since $x \cdot 1 = x$, $0 < x$.
\item What is (mathematically) wrong in the following ``proof'' that $1 = 2$?

  \begin{align*}
    1 &= 2 \\
    0 \cdot 1 &= 0 \cdot 2 \qquad & \text{(Multiply both sides by zero)} \\
    0 &= 0 \qquad & \text{(Any number multiplied by zero is zero)}
  \end{align*}

  It is true that $0 = 0$, therefore we conclude that the original statement $1 = 2$ is also true.

  
\end{enumerate}

\bibliographystyle{amsalpha}
\bibliography{notes}

\end{document}
