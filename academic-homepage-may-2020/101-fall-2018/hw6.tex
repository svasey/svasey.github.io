\documentclass{amsart}
\usepackage{hyperref}
\usepackage{graphicx}

\theoremstyle{definition}
\newtheorem{mydef}{Definition}[section]
\newtheorem{lem}[mydef]{Lemma}
\newtheorem{thm}[mydef]{Theorem}
\newtheorem{cor}[mydef]{Corollary}
\newtheorem{claim}[mydef]{Claim}
\newtheorem{question}[mydef]{Question}
\newtheorem{hypothesis}[mydef]{Hypothesis}
\newtheorem{prop}[mydef]{Proposition}
\newtheorem{defin}[mydef]{Definition}
\newtheorem{example}[mydef]{Example}
\newtheorem{remark}[mydef]{Remark}
\newtheorem{notation}[mydef]{Notation}
\newtheorem{fact}[mydef]{Fact}

\title[Math 101, Fall 2018: assignment 6]{Math 101 - Sets, groups, and topology, Fall 2018 \\ Assignment 6}

%% Include only sections, not subsections, in the table of content.
\setcounter{tocdepth}{1}

%% \author{Sebastien Vasey}
%% \email{sebv@cmu.edu}
%% \address{Department of Mathematical Sciences, Carnegie Mellon University, Pittsburgh, Pennsylvania, USA}
\date{\today}

\begin{document}

%% No indentation at the start of each paragraph
%\parindent 0pt

\maketitle

\textbf{Due Friday, September 28 at the beginning of class} (please submit your assignment on Canvas). Make sure to include your full name \emph{and the list of your collaborators} (if any) with your assignment. You may discuss problems with others, but you may \emph{not} keep a written record of your discussions. Please refer to the syllabus for details.

In your proofs, you may assume basic facts about real numbers and integers, such as those mentioned on p.~90-91 of Hammack. You also can use results from the book, results covered in class, or results you have proven already. As a general rule, you should imagine that you are writing your proof to convince somebody else in the class who is very skeptical about the particular statement. In particular, it should be completely understandable to another student: always justify your reasoning in plain English. 

\begin{enumerate}
\item Write the relation ``divides'' on $A = \{1,2,3,4,5,6\}$ as a set of ordered pairs, then draw a picture of it (as in p.~177 of Hammack).
\item Assume $A = \{a, b, c\}$ and $B = \{d, e, f\}$ (we think of $a, b, c, d, e, f$ as distinct letters of the alphabet here). For each of the relation $R$ from $A$ to $B$ below, first draw a picture of the relation (as in p.~195 of Hammack), then say whether $R$ is function from $A$ to $B$. If you think it is,  justify briefly. If you think it is not, explain exactly where the problem is.

  \begin{enumerate}
  \item $R = \{(a, d), (b, e), (b, e), (c, f)\}$.
  \item $R = \{(a, d), (b, d), (c, f)\}$.
  \item $R = \{(a, d), (a, e), (b, e), (c, f)\}$.
  \item $R = \{(a, f), (b, d), (b, d)\}$.
  \item $R = \emptyset$.
  \end{enumerate}
\item Assume $f: \mathbb{R} \to \mathbb{R}$ is a function and $a$ and $b$ are real numbers. Prove or disprove (\emph{Note: as usual, to disprove it is enough to give examples of a function $f$ and real numbers $a$ and $b$ failing the statement.}):
  \begin{enumerate}
  \item If $a \neq b$, then $f (a) \neq f (b)$.
  \item If $f(a) \neq f (b)$, then $a \neq b$.
  \end{enumerate}
\item Assume $A$, $B$, and $C$ are sets. Assume $R$ is a relation from $A$ to $B$ and $S$ is a relation from $B$ to $C$. The \emph{composition} of $R$ and $S$ (written $S \circ R$) is a relation from $A$ to $C$ defined by:

  $$
  S \circ R = \{(a, c) \in A \times C \mid \text{there exists } b \in B \text{ so that } a R b \text{ and } b S c\}
  $$

  \begin{enumerate}
  \item Assume $A = \{1, 2, 3\}$, $B = \{2, 4, 6\}$, $C = \{7,8, 9\}$, $R = \{(1, 2), (2, 2), (3, 6)\}$, $S = \{(2, 7), (2, 8), (4,9)\}$. Draw a picture of $R$ and $S$ separately, then write $S \circ R$ explicitly as a set of ordered pairs and draw a picture of $S \circ R$.
  \item Explain why the composition of two functions is a function (no need to give a full proof, but give a careful explanation).
  \end{enumerate}

\item Assume $a, b, c$, and $d$ are four ``things'' (for example, four real numbers).
  \begin{enumerate}
  \item If $\{a, b\} = \{c, d\}$, does this mean that $a = c$? What if we also knew that $b = d$?
  \item Prove that if $\{\{a\}, \{a, b\}\} = \{\{c\}, \{c, d\}\}$, then $a = c$ and $b = d$. Explain why this allows us to ``code'' ordered pairs using sets. \emph{Hint: do not forget that it could be for example that $a = b$}.
  \end{enumerate}
\item Assume $A, B, C, $ and $D$ are sets. Prove or disprove:

  \begin{enumerate}
    \item $(A \times B) \cup (C \times D) = (A \cup C) \times (B \cup D)$.
    \item $(A \times B) \cap (C \times D) = (A \cap C) \times (B \cap D)$.
  \end{enumerate}
\item Assume $A$ and $B$ are subsets of some universal set $U$.
  \begin{enumerate}
  \item Explain why it is \emph{not} always true that $(A \times B)^c = A^c \times B^c$. \emph{Note: when taking complements, we think of the universal set for $A \times B$ as $U \times U$. Thus $(A \times B)^c = (U \times U) - (A \times B)$ but $A^c = U - A$, $B^c = U - B$.}
  \item Give a correct formula for $(A \times B)^c$ (involving $A$, $A^c$, $B$, and $B^c$) and prove it. \emph{Hint: you may want to draw a picture when $U = \mathbb{R}$ and $A = [0,1]$, $B = [0,2]$.}
  \end{enumerate}
\item Assume $A$ and $B$ are sets. Prove or disprove:
  \begin{enumerate}
  \item $\mathcal{P} (A \times B) \subseteq \mathcal{P} (A) \times \mathcal{P} (B)$.
  \item $\mathcal{P} (A) \times \mathcal{P} (B) \subseteq \mathcal{P} (A \times B)$.
  \item $\mathcal{P} (A \times B) \subseteq \{A_0 \times B_0 \mid (A_0, B_0) \in \mathcal{P} (A) \times \mathcal{P} (B)\}$.
  \item $\{A_0 \times B_0 \mid (A_0, B_0) \in \mathcal{P} (A) \times \mathcal{P} (B)\} \subseteq \mathcal{P} (A) \times \mathcal{P} (B)$.
  \end{enumerate}
\end{enumerate}

\bibliographystyle{amsalpha}
\bibliography{notes}

\end{document}
