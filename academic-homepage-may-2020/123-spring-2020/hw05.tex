\documentclass{amsart}
\usepackage[utf8]{inputenc}
\usepackage{hyperref}
\usepackage{graphicx}
\usepackage{amssymb}
\usepackage{subfigure}

\theoremstyle{definition}
\newtheorem{mydef}{Definition}[section]
\newtheorem{lem}[mydef]{Lemma}
\newtheorem{thm}[mydef]{Theorem}
\newtheorem{cor}[mydef]{Corollary}
\newtheorem{claim}[mydef]{Claim}
\newtheorem{question}[mydef]{Question}
\newtheorem{hypothesis}[mydef]{Hypothesis}
\newtheorem{prop}[mydef]{Proposition}
\newtheorem{defin}[mydef]{Definition}
\newtheorem{example}[mydef]{Example}
\newtheorem{remark}[mydef]{Remark}
\newtheorem{notation}[mydef]{Notation}
\newtheorem{fact}[mydef]{Fact}

\newcommand{\Ls}{\mathcal{L}}

\newcommand{\Bb}{\mathcal{B}}
\newcommand{\Ps}{\mathcal{P}}

\newcommand{\Zz}{\mathbb{Z}}
\newcommand{\Qq}{\mathbb{Q}}
\newcommand{\Rr}{\mathbb{R}}
\newcommand{\Cc}{\mathbb{C}}

\newcommand{\Ff}{\mathbb{F}}
\newcommand{\Tor}{\operatorname{Tor}}
\newcommand{\Hom}{\operatorname{Hom}}
\newcommand{\Ann}{\operatorname{Ann}}

\title[Math 123, Spring 2020: assignment 5]{Math 123 - Algebra II - Spring 2020 \\ Assignment 5}

%% Include only sections, not subsections, in the table of content.
\setcounter{tocdepth}{1}

\date{\today}

\begin{document}

%% No indentation at the start of each paragraph
%\parindent 0pt

\vspace*{-10em}
\maketitle

\textbf{Due Tuesday, March 3, 11h59pm.} (please submit your assignment as a PDF on Canvas). Unless otherwise noted, references are to Dummit and Foote, \emph{Algebra}, 3rd edition, Wiley, 2004.

Remember that in this class ``ring'' means ``ring with identity''. Everywhere below, $R$ denotes a ring.

\section*{Practice problems} (\emph{Not} for credit. No need to submit your solutions. Try to do them while minimally looking at the textbook)

\begin{enumerate}
\item Give the definition of the following: external direct sum, internal direct sum, free module, basis of a module, rank of a module, Noetherian, elementary divisors, invariant factors, Jordan canonical form.
\item State and prove the structure theorem for modules seen in class (both forms).
\item For what kind of integral domains is a submodule of a free module free, and why?
\item Prove that any square matrix of complex numbers has a Jordan canonical form.
\item True or false?
  \begin{enumerate}
  \item A module is Noetherian if and only if it is finitely generated.
  \item Any abelian group is isomorphic to a product of cyclic groups.
  \item For finitely generated modules over PIDs, free is equivalent to torsion-free.
  \item For $R$ an integral domain, any free $R$-module of rank $n$ is isomorphic to $R^n$.
  \item For $R$ an integral domain, any submodule of a finitely generated free module is finitely generated.
  \end{enumerate}
\end{enumerate}

\section*{Problems for credit}

\begin{enumerate}
\item You should have the assignment from another student available for review in your Canvas todo list. Review problem 3 from that assignment (submit your comment on Canvas, \emph{not} with this assignment). Refer to the peer review instructions on the course website for more details. \emph{You are encouraged not to look at the official solution before submitting your review!}
\item 
  \begin{enumerate}
  \item Let $M_1, M_2$ be $R$-modules and $N_1, N_2$ be submodules of $M_1$ and $M_2$ respectively. Show that $(M_1 \times M_2) / (N_1 \times N_2) \cong (M_1 / N_1) \times (M_2 / N_2)$.
  \item Let $R$ be an integral domain and let $M$ be an $R$-module. Let $y \in M$ and $a \in R$. Assume that $y$ is not a torsion element. Show that $(R y) / (R ay) \cong R / (a)$. Give an example showing that this may no longer hold if $y$ is a torsion element.
  \item Assume $R$ is a commutative ring with $0 \neq 1$. Show that if $R^n \cong R^m$ (as $R$-modules), then $n = m$. \emph{Hint: take a maximal ideal of $R$, and quotient by a suitable submodule. Be careful: we are NOT asuming that $R$ is an integral domain here, so you cannot use the proposition about the rank proven in class.}
  \end{enumerate}
\item
  \begin{enumerate}
  \item (DF, 10.1.9) If $N$ is a submodule of $M$, recall that the \emph{annihilator of $N$ in $R$}, $\Ann (N)$, is defined to be $\{r \in R \mid rn = 0 \text{ for all } n \in N\}$. Show that the annihilator of $N$ in $R$ is an ideal of $R$.
  \item (DF, 10.1.10) If $I$ is an ideal of $R$, the \emph{annihilator of $I$ in $M$} is defined to be $\{m \in M \mid rm = 0 \text{ for all } r \in I\}$. Prove that the annihilator of $I$ in $M$ is a submodule of $M$.
  \item (DF, 10.1.11) Let $M$ be the abelian group $\Zz / 24 \Zz \times \Zz / 15 \Zz \times \Zz / 50 \Zz$, considered as a $\Zz$-module.
    \begin{enumerate}
    \item Give a generator for the annihilator of $M$ in $\Zz$.
    \item Describe the annihilator of $2\Zz$ in $M$ as a product of cyclic groups.
    \end{enumerate}
  \end{enumerate}
\item Assume that $R$ is commutative with $0 \neq 1$. 
  \begin{enumerate}
  \item Prove that all $R$-modules are free if and only if $R$ is a field.
  \item (DF, 10.3.13) Let $F$ be a free $R$-module of finite rank. Prove that $\Hom_R (F, R) \cong F$. Here, $\Hom_R (F, R)$ is the $R$-module of all homomorphisms from $F$ to $R$, under the natural action and addition.
  \end{enumerate}
\item
  \begin{enumerate}
  \item Describe all abelian groups of order 72.
  \item For each of the following abelian groups, determine whether it is finitely generated. If it is, give its free rank, elementary divisors, and invariant factors. We write $\Zz_n$ for $\Zz / n \Zz$.
    \begin{enumerate}
    \item $\Zz \oplus \Zz_6 \oplus \Zz_{16} \oplus \Zz$.
    \item $\Qq$, with addition.
    \item $\Zz_4 \oplus \Zz_4 \oplus \Zz_3$.
    \end{enumerate}
  \end{enumerate}
\item (DF, 12.1.8) Let $R$ be a PID, let $M$ be a (not necessarily finitely generated!) torsion $R$-module, and let $p$ be a prime in $R$. Show that if there exists a nonzero $m \in M$ such that $pm = 0$, then $\Ann (M)  \subseteq (p)$.
\item (DF, 12.1.15) Show that if $R$ is a Noetherian ring, then $R^n$ is a Noetherian $R$-module.
\end{enumerate}
\end{document}
