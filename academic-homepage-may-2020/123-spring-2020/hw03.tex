\documentclass{amsart}
\usepackage[utf8]{inputenc}
\usepackage{hyperref}
\usepackage{graphicx}
\usepackage{amssymb}
\usepackage{subfigure}

\theoremstyle{definition}
\newtheorem{mydef}{Definition}[section]
\newtheorem{lem}[mydef]{Lemma}
\newtheorem{thm}[mydef]{Theorem}
\newtheorem{cor}[mydef]{Corollary}
\newtheorem{claim}[mydef]{Claim}
\newtheorem{question}[mydef]{Question}
\newtheorem{hypothesis}[mydef]{Hypothesis}
\newtheorem{prop}[mydef]{Proposition}
\newtheorem{defin}[mydef]{Definition}
\newtheorem{example}[mydef]{Example}
\newtheorem{remark}[mydef]{Remark}
\newtheorem{notation}[mydef]{Notation}
\newtheorem{fact}[mydef]{Fact}

\newcommand{\Ls}{\mathcal{L}}

\newcommand{\Bb}{\mathcal{B}}
\newcommand{\Ps}{\mathcal{P}}

\newcommand{\Zz}{\mathbb{Z}}
\newcommand{\Qq}{\mathbb{Q}}
\newcommand{\Rr}{\mathbb{R}}
\newcommand{\Cc}{\mathbb{C}}

\title[Math 123, Spring 2020: assignment 3]{Math 123 - Algebra II - Spring 2020 \\ Assignment 3}

%% Include only sections, not subsections, in the table of content.
\setcounter{tocdepth}{1}

\date{\today}

\begin{document}

%% No indentation at the start of each paragraph
%\parindent 0pt

\vspace*{-10em}
\maketitle

\textbf{Due Tuesday, February 18, 11h59pm.} (please submit your assignment as a PDF on Canvas). Unless otherwise noted, references are to Dummit and Foote, \emph{Algebra}, 3rd edition, Wiley, 2004.

Remember that in this class ``ring'' means ``ring with identity''.

\section*{Practice problems} (\emph{Not} for credit. No need to submit your solutions. Try to do them while minimally looking at the textbook)

\begin{enumerate}
\item Give the definition of the following: maximal ideal, prime ideal, direct product of two rings, $I + J$, $I \cdot J$, field of fractions, Euclidean domain, unique factorization domain, irreducible, prime, associates. 
\item State and prove the Chinese remainder theorem for commutative rings.
\item Prove that in a commutative ring, an ideal is prime if and only if the quotient is an integral domain.
\item Prove that every Euclidean domain is a principal ideal domain.
\item Prove  that every principal ideal domain is a unique factorization domain.
\item True or false?
  \begin{enumerate}
  \item In an integral domain, every irreducible element is prime.
  \item In a principal ideal domain, every prime element is irreducible.
  \item Being associate is an equivalence relation.
  \item In $\Zz[\sqrt{-5}]$, $2$ is prime.
  \item In $\Qq[x]$, the only associates of $x^2$ are $x^2$ and $-x^2$.
  \end{enumerate}
\end{enumerate}

\section*{Problems for credit}

\begin{enumerate}
\item You should have the assignment from another student available for review in your Canvas todo list. Review problem 3 from that assignment (submit your comment on Canvas, \emph{not} with this assignment). Refer to the peer review instructions on the course website for more details. \emph{You are encouraged not to look at the official solution before submitting your review!}
\item (DF, 7.4.37) A commutative ring $R$ is called a \emph{local ring} if it has a unique maximal ideal.
  \begin{enumerate}
  \item Prove that if $R$ is a local ring with maximal ideal $M$, then every element of $R - M$ is a unit.
  \item Conversely, prove that if $R$ is a commutative ring in which the set of nonunits forms an ideal $M$, then $R$ is a local ring with unique maximal ideal $M$.
  \end{enumerate}
\item (DF, 7.5.3) Let $F$ be a field. Prove that $F$ contains a unique smallest subfield $F_0$, which is isomorphic to either $\Qq$ or $\Zz / p\Zz$ for some prime $p$.
\item Let $R$ and $S$ be rings.

  \begin{enumerate}
  \item (DF, 7.6.3) Prove that any ideal of $R \times S$ is of the form $I \times J$, for $I$ an ideal of $R$ and $J$ an ideal of $S$.
  \item (DF, 7.6.4) Show that if $R$ and $S$ are not zero rings, $R \times S$ is not a field.
  \end{enumerate}
\item (DF, 8.2.3) Show that the quotient of a PID by a prime ideal is again a PID. Is it still true that the quotient of a PID by an arbitrary ideal is a PID?
\item (DF, 8.2.7) An integral domain $R$ in which every ideal generated by two elements is principal is called a \emph{Bezout domain}.
  \begin{enumerate}
  \item Prove that the integral domain $R$ is a Bezout domain if and only if every pair of elements $a$, $b$ of $R$ has a greatest common divisor\footnote{A \emph{greatest common divisor} of $a$ and $b$ is an element $d$ such that $(a, b) \subseteq (d)$ and for any principal ideal $I$ containing $(a, b)$, $(d) \subseteq I$.} that can be written as $ax + by$ for some $x, y \in R$.
  \item Let $F$ be the field of fractions of a Bezout domain $R$. Prove that every element of $F$ can be written in the form $\frac{a}{b}$, with $a, b \in R$ and $a, b$ relatively prime (that is, with $1$ as a greatest common divisor).
  \end{enumerate}
\item (DF, 8.3.11) Prove that a ring $R$ is a PID if and only if it is a UFD which is also a Bezout domain.
\end{enumerate}
\end{document}
