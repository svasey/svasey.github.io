\documentclass{amsart}
\usepackage[utf8]{inputenc}
\usepackage{hyperref}
\usepackage{graphicx}
\usepackage{amssymb}
\usepackage{subfigure}

\theoremstyle{definition}
\newtheorem{mydef}{Definition}[section]
\newtheorem{lem}[mydef]{Lemma}
\newtheorem{thm}[mydef]{Theorem}
\newtheorem{cor}[mydef]{Corollary}
\newtheorem{claim}[mydef]{Claim}
\newtheorem{question}[mydef]{Question}
\newtheorem{hypothesis}[mydef]{Hypothesis}
\newtheorem{prop}[mydef]{Proposition}
\newtheorem{defin}[mydef]{Definition}
\newtheorem{example}[mydef]{Example}
\newtheorem{remark}[mydef]{Remark}
\newtheorem{notation}[mydef]{Notation}
\newtheorem{fact}[mydef]{Fact}

\newcommand{\Ls}{\mathcal{L}}

\newcommand{\Bb}{\mathcal{B}}
\newcommand{\Ps}{\mathcal{P}}

\newcommand{\Zz}{\mathbb{Z}}
\newcommand{\Qq}{\mathbb{Q}}
\newcommand{\Rr}{\mathbb{R}}
\newcommand{\Cc}{\mathbb{C}}

\newcommand{\Ff}{\mathbb{F}}
\newcommand{\Tor}{\operatorname{Tor}}
\newcommand{\Hom}{\operatorname{Hom}}
\newcommand{\Ann}{\operatorname{Ann}}

\title[Math 123, Spring 2020: assignment 8]{Math 123 - Algebra II - Spring 2020 \\ Assignment 8}

%% Include only sections, not subsections, in the table of content.
\setcounter{tocdepth}{1}

\date{\today}

\begin{document}

%% No indentation at the start of each paragraph
%\parindent 0pt

\vspace*{-10em}
\maketitle

\textbf{Due Tuesday, March 31, 11h59pm.} (please submit your assignment as a PDF on Canvas). Unless otherwise noted, references are to Dummit and Foote, \emph{Algebra}, 3rd edition, Wiley, 2004.

\section*{Practice problems} (\emph{Not} for credit. No need to submit your solutions. Try to do them while minimally looking at the textbook)

\begin{enumerate}
\item Give the definition of the following: constructible real number, splitting field, $n$th root of unity, cyclotomic field of $n$th roots of unity, algebraically closed, algebraic closure, derivative of a polynomial, separable polynomial.
\item State precisely and prove the impossibility of the three classical greek problems: doubling the cube, squaring the circle, trisecting the angle.
\item Prove that splitting fields exist and are unique.
\item Prove that the degree of the splitting field of a polynomial of degree at most $n$ is bounded by $n!$.
\item Prove that the degree of the cyclotomic field of $p$th roots of unity, $p$ prime, is $p - 1$.
\item Prove that a polynomial is separable if and only if it is coprime with its derivative.
\item True or false?
  \begin{enumerate}
  \item Every irreducible polynomial is separable.
  \item Every separable polynomial is irreducible.
  \item Every field has an algebraic closure. This algebraic closure is algebraically closed and unique up to isomorphism.
  \item For any angle $\theta$, if $\theta$ can be constructed then $\theta / 3$ cannot be constructed.
  \item For any angle $\theta$, if $\theta$ can be constructed, then $\theta / 4$ can be constructed.
  \end{enumerate}
\end{enumerate}
\section*{Problems for credit}
\begin{enumerate}
\item You should have the assignment from another student available for review in your Canvas todo list. Review problem 6 from that assignment (submit your comment on Canvas, \emph{not} with this assignment). Refer to the peer review instructions on the course website for more details. \emph{You are encouraged not to look at the official solution before submitting your review!}
\item For the question below, you may use the lemma about constructibility of parallel and perpendicular lines stated in the slides. Use the definitions given in the slides (not in Dummit and Foote).
  \begin{enumerate}
  \item Show that a point $(x, y)$ is constructible if and only if both $x$ and $y$ are constructible.
  \item Show that a circle is constructible if and only if its center is a constructible point and its radius is a constructible number.
  \end{enumerate}
\item (DF, 13.4.1-3) For each of the polynomials below, determine its splitting field and the degree of this splitting field over $\Qq$.
  \begin{enumerate}
  \item $x^4 - 2$.
  \item $x^4 + 2$.
  \item $x^4 + x^2 + 1$.
  \end{enumerate}
\item Show that any algebraically closed field is infinite.
\end{enumerate}
\end{document}
