\documentclass{amsart}
\usepackage[utf8]{inputenc}
\usepackage{hyperref}
\usepackage{graphicx}
\usepackage{amssymb}
\usepackage{subfigure}

\theoremstyle{definition}
\newtheorem{mydef}{Definition}[section]
\newtheorem{lem}[mydef]{Lemma}
\newtheorem{thm}[mydef]{Theorem}
\newtheorem{cor}[mydef]{Corollary}
\newtheorem{claim}[mydef]{Claim}
\newtheorem{question}[mydef]{Question}
\newtheorem{hypothesis}[mydef]{Hypothesis}
\newtheorem{prop}[mydef]{Proposition}
\newtheorem{defin}[mydef]{Definition}
\newtheorem{example}[mydef]{Example}
\newtheorem{remark}[mydef]{Remark}
\newtheorem{notation}[mydef]{Notation}
\newtheorem{fact}[mydef]{Fact}

\newcommand{\Ls}{\mathcal{L}}

\newcommand{\Bb}{\mathcal{B}}
\newcommand{\Ps}{\mathcal{P}}

\title[Math 155r, Fall 2019: assignment 11]{Math 155r - Combinatorics, Fall 2019 \\ Assignment 11}

%% Include only sections, not subsections, in the table of content.
\setcounter{tocdepth}{1}

\date{\today}

\begin{document}

%% No indentation at the start of each paragraph
%\parindent 0pt

\vspace*{-10em}
\maketitle

\textbf{Due Tuesday, November 19 at the beginning of class.} (please submit your assignment as a PDF on Canvas). Make sure to include your full name \emph{and the list of your collaborators} (if any) with your assignment. You may discuss problems with others, but you may \emph{not} keep a written record of your discussions. You may also freely look at the hints at the end of MN\footnote{Matoušek and Nešetřil, \emph{Invitation to discrete mathematics}, 2nd edition, Oxford University Press, 2008.}. Please refer to the syllabus for details.

As a general rule, imagine that you are writing your solution to convince somebody else in the class who is very skeptical about the particular statement. In particular, it should be completely understandable to another student: always justify your reasoning in plain English. It is not sufficient to simply state a number or formula without providing the steps and reasoning that you used to produce the answer.

\begin{enumerate}
\item (MN, 10.2.5) Give an example of a probability space $(\Omega, P)$ and of three events $A, B, C$ in this space so that any two of them are independent, but the three of them together are not independent.
\item (MN, 10.2.3) Prove that a random graph almost surely contains a triangle (see MN, 10.2.4, 10.2.5 for the definitions). 
\item Submit your project draft separately on Canvas. Please note that another student in the class will peer-review your draft. If you would be interested in giving a short in-class presentation of your project, please let me know.
\end{enumerate}



\end{document}
