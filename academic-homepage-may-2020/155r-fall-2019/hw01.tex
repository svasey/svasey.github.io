\documentclass{amsart}
\usepackage[utf8]{inputenc}
\usepackage{hyperref}
\usepackage{graphicx}
\usepackage{amssymb}

\theoremstyle{definition}
\newtheorem{mydef}{Definition}[section]
\newtheorem{lem}[mydef]{Lemma}
\newtheorem{thm}[mydef]{Theorem}
\newtheorem{cor}[mydef]{Corollary}
\newtheorem{claim}[mydef]{Claim}
\newtheorem{question}[mydef]{Question}
\newtheorem{hypothesis}[mydef]{Hypothesis}
\newtheorem{prop}[mydef]{Proposition}
\newtheorem{defin}[mydef]{Definition}
\newtheorem{example}[mydef]{Example}
\newtheorem{remark}[mydef]{Remark}
\newtheorem{notation}[mydef]{Notation}
\newtheorem{fact}[mydef]{Fact}

\title[Math 155r, Fall 2019: assignment 1]{Math 155r - Combinatorics, Fall 2019 \\ Assignment 1}

%% Include only sections, not subsections, in the table of content.
\setcounter{tocdepth}{1}

\date{\today}

\begin{document}

%% No indentation at the start of each paragraph
%\parindent 0pt

\maketitle

\textbf{Due Tuesday, September 10 at the beginning of class} (please submit your assignment as a PDF on Canvas). Make sure to include your full name \emph{and the list of your collaborators} (if any) with your assignment. You may discuss problems with others, but you may \emph{not} keep a written record of your discussions. Please refer to the syllabus for details.

As a general rule, imagine that you are writing your solution to convince somebody else in the class who is very skeptical about the particular statement. In particular, it should be completely understandable to another student: always justify your reasoning in plain English. It is not sufficient to simply state a number or formula without providing the steps and reasoning that you used to produce the answer.

\begin{enumerate}
\item (Extra credit: 15\%) \begin{enumerate}
\item Please fill in the survey at: \\
  \url{http://math.harvard.edu/~sebv/155r-fall-2019/questionnaire.odt}. Submit it separately on Canvas.
\item I like to know my students as human beings, so I would like to have a short one on one 5-10 min chat with you during the first few weeks of the semester, just so that I can know your face, name, and a little bit about your background. Don't be afraid, we're not going to talk math (unless you really want to!). You don't need to prepare anything for the meeting.

  Please send me a short email at \url{sebv@math.harvard.edu} with subject ``155r short meeting'' and ask e.g.\ ``is 1pm next Monday okay?''. I will either reply yes or propose another time. The meeting will take place in my office, SC 321H.
\end{enumerate}
\item (MN, ex.\ 1.2.6)\footnote{References are to Matoušek and Nešetřil, \emph{Invitation to discrete mathematics}, 2nd edition, Oxford University Press, 2008.} Prove carefully (by showing both inclusions) that for any two sets $A$ and $B$, $(A \backslash B) \cup (B \backslash A) = (A \cup B) \backslash (A \cap B)$.
\item   (MN, ex.\ 1.3.5) In ancient Egypt, fractions were written as sums of fractions with numerator 1. For example, $\frac{3}{5} = \frac{1}{2} + \frac{1}{10}$. Consider the following algorithm for writing a fraction $\frac{m}{n}$ in this form ($1 \le m < n$): write the fraction $\frac{1}{\lceil n / m \rceil}$, calculate the fraction $\frac{m}{n} - \frac{1}{\lceil n / m \rceil}$, and if it is nonzero repeat the same step. Prove that the algorithm always terminates in a finite number of steps. \emph{Hint: show that the numerator strictly decreases at each step.}
\item (MN, ex.\ 1.3.9) At time zero, a particle resides at the point zero on the real line. Within one second, it divides into two particles that fly in opposite directions and stop at distance exactly one from where the original particle was (so at time one, there are now two particles). Within the next second, each of the two particles again divides into two particles that fly in opposite directions and stop at distance one from the point of division. Whenever two particles meet, they annhilate and nothing is left. How many particles will there be at time $2^{11} + 1$?
\item Show that at a party with six students, there are three of them that either all know or all do not know each other (we assume that knowledge is a symmetric relation). Show that there is a party of five students where this conclusion fails.
\end{enumerate}



\end{document}
